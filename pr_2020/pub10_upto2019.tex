\documentclass[11pt,twoside,a4paper]{article}
\usepackage{graphicx} 
\usepackage{xcolor}
\usepackage{xspace}
\usepackage{url}
\usepackage{amsmath}
\usepackage[T1]{fontenc}
\usepackage{hyperref}
\newcommand{\HRule}{\rule{\linewidth}{0.2mm}}
\thispagestyle{empty}
\setlength{\parindent}{0mm}
\setlength{\parskip}{0mm}
%%%%%%%%%%%%%%%%%%%%%%%%%%%
\hoffset = -2.0 cm
\voffset = -3.0 cm
\textheight = 25.0 cm
\marginparwidth = 1.0 cm
\evensidemargin = 1.0 cm
\textwidth =16.1 cm
\renewcommand{\arraystretch}{1.5}
%%%%%%%%%%%%%%%%%%%%%%%%%%%

%%% remove "References" title
\usepackage{etoolbox}
\patchcmd{\thebibliography}{\section*{\refname}}{}{}{}
%%%%%%%%%%%%%%%%%%%%%%%%%%%%%%


\begin{document}

\vskip 1.0 truecm
\begin{center}
{\bf{Elenco dei 10 prodotti pi\`u significativi}}
\end{center}
\vskip 0.5 truecm

\begin{enumerate}

\item I.~A.~Costa et al., (CYGNO Collaboration) ``Performance of
  optically readout GEM-based TPC with a $^{55}$Fe
  source''. \textit{JINST} \textbf{14} (07011), 2019. DOI:
  \href{https://iopscience.iop.org/article/10.1088/1748-0221/14/07/P07011}{10.1088/1748-0221/14/07/p07011}.

  Ricostruzione degli eventi, con un clustering basato su algoritmi
  di machine learning non supervisionato.  Questi sono necessari per
  distinguere diverse tipologie di interazioni, come fotoni di
  energia di 5.9 keV, da fondi di raggi cosmici e radioattivit\`a
  naturale. Analisi dei dati raccolti con diversi prototipi.

\item A.~Nisati, V.~Sharma et al., ``Discovery of the Higgs
  Boson''. \textit{World Scientific} \textbf{ISBN:
    978-981-4425-44-5}, 2017. DOI:
  \href{https://www.worldscientific.com/worldscibooks/10.1142/8595}{10.1142/8595}.

  Editor del capitolo sul canale $H\to WW\to\ell^+\nu\ell^-\nu$,
  revisione di tutti i capitoli, contributi ai capitoli sulla
  combinazione dei diversi canali e dei capitoli generali.

\item V.~Khachatryan et al. (CMS Collaboration), ``Performance of
  Electron Reconstruction and Selection with the CMS Detector in
  Proton-Proton Collisions at $\sqrt{s} = 8$ TeV''. \textit{JINST}
  \textbf{10} (P06005), 2015. DOI:
  \href{https://iopscience.iop.org/article/10.1088/1748-0221/10/06/P06005}{10.1088/1748-0221/10/06/P06005}
  
  Ricostruzione degli elettroni, miglioramento della risoluzione in
  momento con l'applicazione di una regressione multivariata che
  combina l'impulso della traccia e l'energia nel
  calorimetro. Integrazione nella ricostruzione globale dell'evento
  di CMS ``particle-flow''. Calibrazione della scala di energia con
  risonanze Z, $J/\psi$ e $\Upsilon$, per coprire un intervallo di
  impulso rilevante per il decadimento del bosone di Higgs in 4
  leptoni carichi ($\approx$5--50 GeV).

\item S.~Chatrchyan et al. (CMS Collaboration), ``Energy Calibration
  and Resolution of the CMS Electromagnetic Calorimeter in pp
  Collisions at $\sqrt{s} = 7$ TeV''. \textit{JINST} \textbf{8}
  (P09009), 2013. DOI:
  \href{https://iopscience.iop.org/article/10.1088/1748-0221/8/09/P09009/meta}{10.1088/1748-0221/8/09/P09009}

  Commissioning del rivelatore ECAL, con l'installazione, la
  calibrazione e l'ottimizzazione del sistema di alta tensione per gli
  APD durante la presa dati. Upgrade del laser per il controllo della
  trasparenza da un sistema \textit{lamp-pumped} a uno di nuova
  generazione \textit{diode-pumped} a stato solido. Sviluppo del DQM e
  suo utilizzo per l'analisi dei dati di basso livello durante il
  periodo iniziale di collisioni.

\item S.~Chatrchyan et al. (CMS Collaboration), ``Energy Calibration
  and Resolution of the CMS Electromagnetic Calorimeter in pp
  Collisions at $\sqrt{s} = 7$ TeV''. \textit{JINST} \textbf{8}
  (P09009), 2013. DOI:
  \href{https://iopscience.iop.org/article/10.1088/1748-0221/8/09/P09009/meta}{10.1088/1748-0221/8/09/P09009}

  Commissioning del rivelatore ECAL, con l'installazione, la
  calibrazione e l'ottimizzazione del sistema di alta tensione per gli
  APD durante la presa dati. Upgrade del laser per il controllo della
  trasparenza da un sistema \textit{lamp-pumped} a uno di nuova
  generazione \textit{diode-pumped} a stato solido. Sviluppo del DQM e
  suo utilizzo per l'analisi dei dati di basso livello durante il
  periodo iniziale di collisioni.

\item S.~Chatrchyan et al. (CMS Collaboration), ``Measurement of Higgs
  boson production and properties in the WW decay channel with
  leptonic final states''. \textit{JHEP} \textbf{01} (96), 2014. DOI:
  \href{https://link.springer.com/article/10.1007/JHEP01(2014)096}{10.1007/JHEP01(2014)096}

  Setup dell'analisi, dai primissimi dati di collisione di CMS, fino
  alla misura delle propriet\`a del bosone di Higgs con i dati di
  Run1. Ottimizzazione della selezione, stima di tutti i fondi con
  campioni di controllo, fit con template 2D. Sviluppo di variabili
  cinematiche utilizzate anche nelle ricerche di SUSY, in presenza di
  pi\`u di una particella invisibile nello stato finale. Coordinamento
  del gruppo di fisica.

\item S.~Chatrchyan et al. (CMS Collaboration), ``Measurement of the
  properties of a Higgs boson in the four-lepton final state''.
  \textit{Phys.\,Rev.\,D} \textbf{89} (092007), 2014. DOI:
  \href{https://journals.aps.org/prd/abstract/10.1103/PhysRevD.89.092007}{10.1103/PhysRevD.89.092007}

  Setup dell'analisi, ottimizzazione della selezione e identificazione
  multivariata degli elettroni in un intervallo di $p_T$ fino a
  7\,GeV. Fit per la determinazione degli accoppiamenti e della massa
  del bosone Higgs con l'inclusione dell'incertezza di massa
  evento-per-evento. Sviluppo dell'unfolding con fit 8D per la
  determinazione degli accoppiamenti anomali del bosone di Higgs ai
  bosoni vettori. Editor dell'articolo e coordinamento del sottogruppo
  di analisi.

\item S.~Chatrchyan et al. (CMS Collaboration), ``Observation of a New
  Boson at a Mass of 125 GeV with the CMS Experiment at the LHC''.
  \textit{Phys.\,Lett.\,B} \textbf{716} (30), 2012. DOI:
  \href{https://www.sciencedirect.com/science/article/pii/S0370269312008581}{10.1016/j.physletb.2012.08.021}

  Analisi complete di due dei canali di decadimento maggiormente
  sensibili: $H \to W^+W^-\to \ell^+\nu\ell^-\nu$ e $H\to ZZ\to 4\ell$, con
  stati finali puramente leptonici. Misura dell'accoppiamento ai
  bosoni W e Z con la produzione gluon-fusion, VBF e produzione
  associata di W e Z. Con il canale $H\to4\ell$, prima misura di
  precisione della massa del bosone Higgs.

\item S.~Chatrchyan et al. (CMS Collaboration), ``Jet Production Rates
  in Association with W and Z Bosons in pp collisions at $\sqrt{s}=7$
  TeV''. \textit{JHEP} \textbf{01} (010), 2012. DOI:
  \href{https://link.springer.com/article/10.1007/JHEP01(2012)010}{10.1007/JHEP01(2012)010}

  Setup e ottimizzazione dell'analisi, sviluppo del software di
  analisi dati, identificazione di elettroni e muoni, fit per
  l'unfolding della sezione d'urto differenziale.

\item A.~Aubert et al. (BaBar Collaboration), ``Measurement of CP
  Asymmetries in $B^0 \to K^0_{S} K^0_{S} K^0_{S}$
  Decays''. \textit{Phys.\,Rev.\,D} \textbf{76} (091101), 2007. DOI:
  \href{https://journals.aps.org/prd/abstract/10.1103/PhysRevD.76.091101}{10.1103/PhysRevD.76.091101}

  Setup e ottimizzazione dell'analisi, determinazione del vertice in
  un canale con solo mesoni $K^0_S$ usnado il vincolo sul beam spot
  dei fasci $e^+e^-$, misure di efficienza di ricostruzione dei
  $K^0_S$ e fit di CP dipendente dal tempo.

\item A.~Aubert et al. (BaBar Collaboration), ``Measurements of
  CP-violating asymmetries in the decay $B^0 \to K^{+} K^{-}
  K^0$''. \textit{Phys.\,Rev.\,Lett.} \textbf{99} (161802), 2007. DOI:
  \href{https://journals.aps.org/prl/abstract/10.1103/PhysRevLett.99.161802}{10.1103/PhysRevLett.99.161802}

  Setup e ottimizzazione dell'analisi, ricostruzione e identificazione
  dei $K^0_L$ con una delle prime applicazioni di Machine Lerning in
  fisica delle alte energie (BDT), fit di CP dipendente dal tempo nel
  Dalitz plot del decadimento a tre corpi del mesone $B$.
    
\end{enumerate}

\end{document}
