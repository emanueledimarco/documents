\documentclass[11pt,twoside,a4paper]{article}
\usepackage{geometry}
\geometry{a4paper, top=2.3cm, bottom=2.3cm, left=2.3cm, right=2.3cm}
\usepackage{graphicx} 
\usepackage{eurosym}
\usepackage{xcolor}
\usepackage{xspace}
\usepackage{url}
\usepackage{amsmath}
\usepackage{cite}
\usepackage[T1]{fontenc}
\newcommand{\HRule}{\rule{\linewidth}{0.2mm}}
\thispagestyle{empty}
\setlength{\parindent}{0mm}
\setlength{\parskip}{0mm}
%%%%%%%%%%%%%%%%%%%%%%%%%%%
%% \hoffset = -2.0 cm
%% \voffset = -3.0 cm
%% \textheight = 25.0 cm
%% \marginparwidth = 1.0 cm
%% \evensidemargin = 1.0 cm
%% \textwidth =16.1 cm
%% \renewcommand{\arraystretch}{1.5}
%%%%%%%%%%%%%%%%%%%%%%%%%%%

%%% remove "References" title
\usepackage{etoolbox}
\patchcmd{\thebibliography}{\section*{\refname}}{}{}{}
%%%%%%%%%%%%%%%%%%%%%%%%%%%%%%


\begin{document}

\begin{center}
{\bf{Attivit\`a di ricerca scientifica}}
\end{center}

La mia attivit\`a di ricerca \`e incentrata sulla fisica sperimentale
delle alte energie\footnote{Le citazioni alle pubblicazioni su rivista
si riferiscono all'elenco completo delle pubblicazioni. Le citazioni
delle presentazioni a conferenza si riferisco all'elenco di 20
conferenze scelte.}.


\vskip 1.0 truecm
{\bf{Ricerca scientifica nell'esperimento BaBar}}
\vskip 0.5 truecm

\textit{Analisi dati.}
\vskip 0.2 truecm

Dal 2002 al 2007 (lavoro per la tesi di laurea e di dottorato) la mia
attivit\`a \`e stata concentrata sullo studio della violazione della
simmetria CP nei decadimenti deboli del mesone $B$ e sulla ricerca di
segnali indiretti di fisica oltre il Modello Standard nel settore del
sapore, attraverso i decadimenti \textit{charmless} $b \to s$. In
particolare, sulla misura della violazione di CP dipendente dal tempo
del mesone $B$ in tre mesoni $K$, prima nell'approssimazione
\textit{quasi due corpi}~\cite{Aubert:2008rr} e, infine, includendo
gli effetti di interferenza attraverso l'analisi del Dalitz
plot~\cite{Aubert:2007sd,Aubert:2007me}. Attraverso questo studio \`e
stata scoperta una nuova risonanza di spin zero, con una massa circa
di 1.55 GeV/c$^2$, la $X_0(1550)$, che decade in due kaoni
carichi. Per questi studi ho dovuta sviluppare una tecnica di misura
ad-hoc dei vertici di decadimento dei mesoni $K^0_S$, usando il
vincolo del punto di interazione dei fasci $e^+e^-$ del collisore SLAC
di Stanford (California, USA)~\cite{Aubert:2005dy,Aubert:2005gj}.
Queste misure hanno vincolato in modo significativo la presenza di
processi non previsti dal Modello Standard \textit{Flavor Changing
  Neutral Current}.  Ho effettuato queste misure utilizzando i dati
dell'esperimento BaBar, collocato nel punto di interazione del
collisore $e^+e^-$ di SLAC. Ho presentato i risultati a varie
conferenze~[c1--c5].

\vskip 0.5 truecm
\textit{Misure si servizio per l'intera collaborazione.}
\vskip 0.2 truecm

Dal 2005 al 2006 sono stato responsabile delle misure di efficienza
del \textit{$B$-flavour tagging} e della risoluzione di vertice, usate
dall'intera collaborazione BaBar per tutte le misure di violazione di
CP dipendenti dal tempo. Nel 2005 ho effettuato misure relative al
tracciamento di particelle cariche, all'interno della task force per
il recupero della perdita di efficienza di ricostruzione dei
$K^0_S$. Questo ha permesso di recuperare l'intera efficienza durante
lo shutdown temporaneo dell'acceleratore e prima del nuovo periodo di
presa dati.

\vskip 0.5 truecm
\textit{Attivit\`a sul rivelatore.}
\vskip 0.2 truecm

Dal 2005 al 2006 sono stato \textit{operations manager} per gli studi
di performance per il rivelatore \textit{Instrumented Flux
  Return}~(IFR) di BaBar basato sulla tecnologia RPC, necessario per
la ricostruzione e identificazione dei muoni e degli adroni neutri. In
questo contesto ho personalmente sviluppato algoritmi innovativi per
la ricostruzione di $K^0_L$, basati sull'applicazione di
\textit{Boosted Decision Trees}, una delle prime applicazioni di
\textit{machine learning} nella fisica delle alte energie, allo scopo
di combinare in modo ottimale le informazioni del calorimetro
elettromagnetico e dell'IFR. Durante questo periodo ho anche
effettuato lo studio delle possibili cause di invecchiamento degli RPC
a causa della produzione di acido fluoridrico nella miscela di gas, e
le soluzioni per mitigarne l'effetto~\cite{Band:2008zza}. Con la loro
applicazione, il rivelatore ha mantenuto un'efficienza stabile
attraverso il periodo rimanente di presa dati, anche dopo il
cambiamento di modalit\`a di operazione da regime \textit{streamer} a
\textit{avalanche}~\cite{Band:2006ig,Anulli:2005wi}. Ho partecipato
all'upgrade del settore barrel dell'IFR da RPC a LST.

\clearpage

\vskip 1.0 truecm
{\bf{Ricerca scientifica nell'esperimento CMS}}
\vskip 0.5 truecm

\textit{Attivit\`a legate al rivelatore ECAL.}
\vskip 0.2 truecm

Dal 2007 sono collaboratore dell'esperimento CMS presso l'LHC di
Ginevra, e sono stato coinvolto in numerose attivit\`a sperimentali
sul rivelatore e di analisi dei dati. Dal 2007 al 2010 ho partecipato
al \textit{commissioning} del sistema che fornisce l'alta tensione
agli avalanche photodiodes (APD) del calorimetro elettromagnetico
(ECAL)~\cite{Bartoloni_2013}. Il guadagno nominale di 50 si raggiunge
con un voltaggio di circa 400\,V, e il sistema, una volta calibrato,
ha dimostrato una stabilit\`a migliore dello 0.01\% durante tutto il
Run1 e Run2.  Nello stesso periodo ho anche sviluppato il \textit{data
  quality monitoring} (DQM) di ECAL, ancora oggi in
uso~\cite{DiMarco:2009zz}[c6].

Inoltre, dal 2011 al 2014 sono stato responsabile del sistema hardware
con il laser, usato per il monitoring continuo delle variazioni di
trasparenza dei cristalli di $PbWO_4$, dovute all'irraggiamento
durante i fill di LHC. Durante questo periodo il laser
\textit{lamp-pumped} \`e stato cambiato per evitare salti nella
risposta dei cristalli ad ogni intervento sulla lampada. Ho
partecipato all'installazione e messa in opera del nuovo laser
\textit{diode-pumped} a stato solido. Il nuovo sistema \`e ancora in
uso, ed ha dismostrato una stabilit\`a di intensit\`a meglio del 3\%,
e un jitter inferiore ai 3\,ns, necessario per la sincronizzazione con
LHC.

\vskip 0.5 truecm

Dal 2014 al 2017 sono stato il coordinatore del gruppo di performance
(DPG) dell'ECAL, avendo la responsabilit\`a per l'ottimizzazione delle
condizioni di presa dati, per la ricostruzione e per la calibrazione
dei depositi di energia nel
calorimetro~\cite{Khachatryan:2015iwa}. Durante questo periodo ho
coperto la transizione dal Run1 al Run2, quando la spaziatura
temporale tra i bunch di protoni LHC \`e diminuita da 50 a 25\,ns. Nel
2014 ho sviluppato un algoritmo di ricostruzione dell'ampiezza del
segnale digitizzato in ECAL totalmente innovativo, chiamato
\textit{multifit}, e basato sul fit simultaneo di segnali di
bunch-crossing diversi che si sovrappongono nell'intervallo di tempo
della digitizzazione.  Questo permette di azzerare il peggioramento di
risoluzione energetica dovuto al pileup di eventi di bunch-crossing
vicini a quello nominale.  Questa tecnica, sviluppata durante lo
shutdown, \`e stata applicata nella ricostruzione offline lungo tutto
il corso di Run2, e, nel 2016, estesa anche al trigger di alto
livello.  Essa ha permesso di far rimanere inalterata la risoluzione
energetica di ECAL rispetto agli effetti causati dal pileup, fino a 60
interazioni per incrocio dei fasci, valore massimo durante il Run2.
Il funzionamento dell'algoritmo \`e stato dimostrato su simulazioni
fino a una media di 200 interazioni per incrocio dei fasci, e si
prevede di usarlo per la fase di alta luminosit\`a di LHC: in questo
caso le prestazioni del multifit migliorano ancora, grazie alla
maggiore frequenza di campionamento dell'impulso. L'algoritmo e le sue
performance ottenute sui dati di Run2 sono stati presentati a
conferenza~[c13,c15] e pubblicati sulla rivista \textit{JINST} nel
2020.

\vskip 0.5 truecm
\textit{Ricostruzione e identificazione di elettroni e fotoni.}
\vskip 0.2 truecm

Dal 2008 sono stato uno degli sviluppatori della ricostruzione di
elettroni di CMS~\cite{Khachatryan:2015hwa}, che ha permesso tutte le
misure sui bosoni elettrodeboli W e Z, e le primissime ricerche del
bosone di Higgs nei canali di decadimento pi\`u puri, WW e ZZ negli
stati finali completamente leptonici. Dal 2013 al 2014 sono stato
co-responsabile del sottogruppo di \textit{EGamma} per la
ricostruzione e l'identificazione di elettroni e fotoni di CMS.

\vskip 0.5 truecm
\textit{Analisi dei dati.}
\vskip 0.2 truecm

Con i primissimi dati forniti da LHC all'esperimento CMS (36 pb$^{-1}$
a un'energia $\sqrt{s}=7$\,TeV) sono stato l'autore principale della
prima misura di sezione d'urto di produzione di bosoni W e Z,
inclusiva e differenziale nel numero di jet adronici
associati~\cite{Chatrchyan:2011ne,Marco:2009dvd,Khachatryan:2010xn,CMS:2011aa}[c7--c9].
Con gli stessi dati \`e stato possibile effettuare la prima misura
differenziale dell'asimmetria di carica del bosone W, utile per
vincolare le PDF del protone nei range di momento esplorati da
LHC~\cite{Chatrchyan:2011jz}.

\vskip 0.5 truecm

Fin dall'inizio dell'attivit\`a in CMS sono stato coinvolto nella
ricerca diretta del bosone di Higgs del Modello
Standard~\cite{Chatrchyan:2011tz,Chatrchyan:2012tx,Chatrchyan:2012ty}. In
questo contesto, dal 2011 al 2012 sono stato il coordinatore del
gruppo di analisi per la ricerca del bosone di Higgs in coppie di
bosoni W. Ho condotto personalmente l'analisi dello stato finale
completamente leptonico, che gi\`a con i dati a $\sqrt{s}=7$\,TeV ha
escluso ipotesi di massa superiori a 130 GeV, e dato la prima
indicazione della particella di massa intorno a 125 GeV, e, con il
dataset di Run1 ha contribuito, assieme agli altri canali principali,
alla sua
scoperta~\cite{Chatrchyan:2012ufa,Chatrchyan:2013lba,Chatrchyan:2012ufa}[c10,c11].
Ho presentato i risultati con i dati di Run1 di LHC alla conferenza
ICHEP 2012~[c12]. Ho anche proposto e applicato un metodo di analisi
che permette una stima della massa di una risonanza che decade in uno
stato finale con due neutrini: nel caso di H$\to$WW permette una
risoluzione di massa del 3\%~\cite{Chatrchyan:2013iaa}, che non \`e
competitivo con gli stati finali $\gamma\gamma$ e ZZ (4 leptoni
carichi), ma \`e stato usato per ricerche di particelle
supersimmetriche con catene di decadimento
complesse~\cite{Khachatryan:2016epu}. Allo stesso tempo ho misurato la
sezione d'urto di produzione inclusiva di coppie di bosoni W a
$\sqrt{s}=$ 7 e 8 TeV~\cite{Chatrchyan:2013yaa}. Sono uno degli autori
del libro \textit{``Discovery of the Higgs Boson''}, pubblicato da
World Scientific nel 2016, rivolto a studenti laureati e ricercatori
in fisica delle particelle, sulle misure che hanno portato alla
scoperta del bosone di Higgs~\cite{Nisati:2017cst}.

\vskip 0.5 truecm

Sono stato autore principale ed editor dell'articolo \textit{legacy}
di Run1 per la ricerca del bosone di Higgs nel canale di decadimento H
$\to$ ZZ $\to 4\ell$, riguardante le prime misure di accoppiamento ai
bosoni Z, della massa e delle propriet\`a di spin-CP del bosone
scoperto~\cite{Chatrchyan:2013mxa}. Ho coordinato il gruppo di analisi
di questo canale di decadimento con i dati di Run1 dal 2013 al
2014~\cite{Khachatryan:2014kca}. Ho sviluppato la tecnica di analisi
per effettuare l'unfolding dei parametri di accoppiamento effettivi
del bosone di Higgs attraverso un fit simultaneo 8-dimensionale al set
completo di variabili cinematiche del decadimento in 4 leptoni,
collaborando con un gruppo di fisici teorici~\cite{Chen:2014pia} e
presentato i risultati a conferenza~[c14].

\vskip 0.5 truecm

Nel periodo dal 2015 al 2017 ho partecipato alla ricerca di materia
oscura con i dati dell'esperimento CMS in stati finali con energia
trasversa mancante nell'evento e un singolo jet (\textit{monojet} e
decadimenti in stati finali invisibili del bosone di
Higgs)~\cite{Sirunyan:2017hci,Sirunyan:2017jix}~[c16].

\vskip 0.5 truecm

Dal 2017 al 2020 sono stato l'autore principale delle misure collegate
alla determinazione della massa del bosone W con l'esperimento CMS. In
particolare ho messo a punto la misura della sezione d'urto di
produzione del bosone W, multi-differenziale nell'elicit\`a,
rapidit\`a del W, e inoltre doppio-differenziale rispetto al momento
trasverso e pseudo-rapidit\`a del leptone di decadimento, combinando
gli stati finali con elettrone e muone. La misura \`e di rilevanza nel
programma di misura della massa del bosone W~[c17] che ha l'obiettivo
di raggiungere una precisione $\delta m_W/m_W\approx 10^{-4}$,
poich\'e permette di vincolare in-situ le PDF del protone
nell'intervallo di momento rilevante. La tecnica di unfolding
sperimentale \`e all'avanguardia, avendo sviluppato a tal scopo un
algoritmo di minimizzazione della likelihood alternativo a
\textsc{MINUIT}, basato su \textsc{TENSORFLOW}, che permette di
trattare likelihood molto complesse, con cuspidi locali e un numero di
parametri misurati simultaneamente superiore al migliaio. I risultati
dell'analisi e la descrizione degli strumenti innovativi sono stati
pubblicati sulla rivista \textit{Phys.\,Rev.\,D} nel 2020. La stessa
tecnica di misura e strumenti matematici si stanno usando per la prima
misura della massa del W in CMS, basandosi solamente sulla cinematica
del leptone, allo scopo di ridurre al minimo gli input teorici
esterni, come le PDF e il modello dello spettro in $p_T$ del bosone W.
Ci\`o permetter\`a di minimizzare le sistematiche dominanti di tale
misura ad un collisore pp, quale LHC.

\vskip 0.5 truecm
\textit{Centro di calcolo Tier-2 di Roma.}
\vskip 0.2 truecm

Dal 2017 sono co-responsabile del centro di calcolo (\textit{Tier-2})
per i dati di CMS di Roma, un sistema condiviso con l'esperimento
ATLAS. Esso \`e costituito da dieci rack refrigerati ad acqua, che
contengono qualche centinaio di server di elaborazione dati (per un
totale di circa 2500 core) e archiviazione dei dati (per un totale di
circa 2\,PB). Il centro di calcolo permette l'eleborazione di circa il
20\% dei dati raccolti dal LHC ogni anno, ed \`e connesso al sistema
di \textit{grid} mondiale. Sono stato anche il responsabile della
pianificazione degli upgrade e degli acquisti per il Tier-2 di CMS,
per un investimento medio di circa 100\,k\euro{} all'anno.


\vskip 1.0 truecm
{\bf{Ricerca scientifica nell'esperimento CYGNO}}
\vskip 0.5 truecm

\textit{Ricostruzione e analisi dei dati.}
\vskip 0.2 truecm

Dal 2016 partecipo allo sviluppo di CYGNO~\cite{Pinci:2019ztr}, un
rivelatore per la ricerca diretta di materia oscura di tipo
\textit{Weak Interacting Massive Particle} (WIMP), che si manifesti
nei rinculi delle ipotetiche particelle sui nuclei di atomi di gas,
con energia cinetica di pochi keV. Ho partecipato alla costruzione e
caratterizzazione di diversi prototipi del rivelatore, costituito da
una TPC a gas, in cui rivelatori GEM sono posti nella posizione di
anodo, e, oltre all'amplificazione di carica, producono una luce di
scintillazione secondaria nel processo di valanga, che \`e letta
otticamente da una telecamera con sensore CMOS sensibile al singolo
fotone.  In questo contesto ho sviluppato da zero la ricostruzione
degli eventi, rappresentati dalle immagini registrate dal sensore con
pi\`u di 4$\cdot$10$^6$ pixel.  Essa \`e principalmente un algoritmo
di clustering basato su tecniche avanzate di machine learning non
supervisionato, in grado di ricostruire efficientemente sia i pattern
semplici di depositi di energia di raggi X prodotti da sorgenti
radioattive, quali il $^{55}$Fe~\cite{Costa_2019}, ma anche tracce
pi\`u lunghe e complesse, prodotte da raggi cosmici o dalla
radioattivit\`a naturale.  Con i dati raccolti da diversi prototipi
assemblati ai Laboratori Nazionali di Frascati, \`e stato possibile
caratterizzare il rivelatore con un fascio di elettroni della
\textit{Beam Test Facility}, con le sorgenti radioattive, e con i
raggi cosmici~[c18]. Infine ho personalmente effettuato l'analisi dei
dati con rinculi nucleari di energie cinetiche nell'intervallo di
interesse per candidati WIMP, prodotti da una sorgente di $^{241}$Am
contenuta in una capsula di berillio. I risultati sono stati
pubblicati su rivista nel 2020.

\vskip 0.5 truecm
\textit{Attivit\`a di servizio.}
\vskip 0.2 truecm

Dal 2018 sono il responsabile della ricostruzione e del gruppo di
analisi dati dell'esperimento CYGNO, il progetto che \`e previsto
avere un volume sensibile di gas di circa 1\,m$^3$, che dovrebbe
essere installato ai Laboratori sotterranei del Gran Sasso nei
prossimi anni.  Ho definito e implementato il formato \textit{raw} dei
dati in uscita dall'acquisizione, e quello prodotto dalla
ricostruzione. Quest'ultimo \`e di immediata fruizione per tutta la
collaborazione, inclusi numerosi studenti di laurea e di dottorato di
ricerca in fisica.

\clearpage

\nocite{*}

\bibliographystyle{habbrvyr} % reverse cronological order using habbrvyr.bst
\bibliography{INSPIRE-CiteAll-2004-2019}

\end{document}
