\documentclass[11pt,twoside,a4paper]{article}
\usepackage{geometry}
\geometry{a4paper, top=2.3cm, bottom=2.3cm, left=2.3cm, right=2.3cm}
\usepackage{graphicx} 
\usepackage{marvosym}
\usepackage{xcolor}
\usepackage{xspace}
\usepackage{url}
\usepackage{amsmath}
\usepackage{cite}
\usepackage[T1]{fontenc}
\newcommand{\HRule}{\rule{\linewidth}{0.2mm}}
\thispagestyle{empty}
\setlength{\parindent}{0mm}
\setlength{\parskip}{0mm}
%%%%%%%%%%%%%%%%%%%%%%%%%%%
%% \hoffset = -2.0 cm
%% \voffset = -3.0 cm
%% \textheight = 25.0 cm
%% \marginparwidth = 1.0 cm
%% \evensidemargin = 1.0 cm
%% \textwidth =16.1 cm
%% \renewcommand{\arraystretch}{1.5}
%%%%%%%%%%%%%%%%%%%%%%%%%%%

%%% remove "References" title
\usepackage{etoolbox}
\patchcmd{\thebibliography}{\section*{\refname}}{}{}{}
%%%%%%%%%%%%%%%%%%%%%%%%%%%%%%


\begin{document}

\begin{center}
\textbf{Elenco dei contratti o incarichi di ricerca}
\end{center}
\vskip 0.8 truecm

\begin{enumerate}  
\item \textbf{2007--2009. Assegno di ricerca Sapienza Universit\`a di
  Roma e INFN, sezione di Roma}
\vskip 0.3 truecm
Selezione pubblica per titoli e colloquio.
\vskip 0.1 truecm

Nel corso del suo svolgimento ho partecipato al commissioning del
sottorivelatore ECAL di CMS, installando, calibrando e ottimizzando il
setup del sistema di distribuzione dell'alta tensione agli APD del
calorimetro elettromagnetico. Ho sviluppato il Data Quality Monitoring
(DQM) di ECAL, e attraverso di esso ho effettuato l'analisi di basso
livello dei dati del calorimetro con i primissimi dati di LHC.

Ho messo a punto l'analisi per la misura della sezione d'urto di
produzione inclusiva di bosoni W e Z e differenziale nel numero di
jet, e in contmporanea preparato l'analisi per la ricerca del bosone
di Higgs nel canale in 2 bosoni W.

\vskip 0.5 truecm
\item \textbf{2008--2009. Associate presso il CERN (``1$^a$ edizione del
  programma similfellow'' dell'INFN)}
\vskip 0.2 truecm
Selezione basata su titoli, programma di ricerca e
lettere di presentazione, da parte di una commissione nominata dall'INFN.
\vskip 0.1 truecm

Nel corso del suo svolgimento ho sviluppato e ottimizzato il DQM del
sottorivelatore ECAL, e partecipato alle campagne di calibrazione
dell'alta tensione di ECAL. Durante questo periodo sono anche stato il
responsabile e on-call del sistema, e ho affrontato e risolto
l'insorgere di un rumore di pickup dei cavi, una volta che il
rivelatore \`e stato per la prima volta operato con una presa dati
globale con raggi cosmici.

Nello stesso periodo ho lavorato alla preparazione dell'analisi dati
con eventi simulati per la misura della sezione d'urto differenziale
di produzione di bosoni W e Z associati a jet adronici.


\vskip 0.5 truecm
\item \textbf{2009--2011. Assegno di ricerca Sapienza Universit\`a di
  Roma e INFN, sezione di Roma}
\vskip 0.2 truecm
Selezione pubblica per titoli e colloquio.
\vskip 0.1 truecm

Nel corso del suo svolgimento ho svolto l'analisi sui primi dati di
collisione di CMS, che ha prodotto la pubblicazione della sezione
d'urto di produzione inclusiva di bosoni W e Z e differenziale nel
numero di jet associati.

Inoltre ho svolto l'analisi dei dati per la ricerca del bosone di
Higgs H$\to$WW nei canali puramente leptonici: i risultati hanno
portato alla prima esclusione di intervalli di massa intorno a
$m_H\approx$160 GeV con i dati di LHC e le prime evidenze di eccessi
intorno a 125 GeV. Sono stato autore pricincipale dell'articolo,
nonch\'e sono stato convener del gruppo di analisi (``Livello 3'' del
gruppo di fisica dell'Higgs).


\vskip 0.5 truecm
\item \textbf{2011--2014. ``Tolman Prize'' Fellowship  presso Caltech (California Institute of Technology - USA)}
\vskip 0.2 truecm
Selezione per titoli e programma di ricerca.
\vskip 0.1 truecm

Durante il suo svolgimento ho effettuato le misure che hanno portato
alla scoperta del bosone di Higgs del Modello Standard in due dei tre
canali principali: $H \to WW\to 2\ell 2\nu$ (autore e convener del
gruppo) e $H\to ZZ \to 4\ell$ (autore ed editor dell'articolo
principale). A seguito di ci\`o ho presentato i risultati sul primo
canale alla conferenza ICHEP 2012 di Melbourne e partecipato alla
scrittura del libro sulla scoperta del bosone di Higgs, pubblicato da
World Scientific.

Ho anche avuto la responsabilit\`a istituzionale del sistema hardware
per il monitoring della trasparenza di ECAL con laser, e come
rappresentante di Caltech nell'\textit{Institution Board} di ECAL.


\vskip 0.5 truecm
\item \textbf{2014--2017 ``Marie-Curie COFUND'' Fellowship presso il CERN (Ginevra)}
\vskip 0.2 truecm
CERN COFUND \`e stata un'estensione del programma di
Fellowship del CERN, cofinanziato dall'Unione Europea come azione
Marie Curie. La selezione, da parte di un comitato di esperti, si basa
sul curriculum e sul programma di ricerca dei candidati. Le fellowship
COFUND sono attribuite al miglior 10\% dei candidati.
\vskip 0.1 truecm

Durante il suo corso ho principalmente svolto attivit\`a legate al
sottorivelatore ECAL, come convener del gruppo \textit{Detector
  Performance Group} (DPG), responsabile dell'ottimizzazione
dell'efficienza di presa dati e risoluzione energetica del calorimetro
elettromagnetico. Poich\'e la prima parte della responsabilit\`a \`e
coincisa con il periodo di shutdown tra Run1 e Run2, ho sviluppato da
zero una ricostruzione alternativa dell'ampiezza digitizzata da ECAL,
che fosse resiliente al pileup di depositi da bunch-crossings non in
tempo con quello principale. Questo nuovo algoritmo, usato per
l'intero Run2, ha permesso alla risoluzione di ECAL di rimanere
all'eccellente livello di Run1. L'algoritmo \`e stato pubblicato su
\textit{JINST} nel 2020, sar\`a in uso anche per il Run3, ed \`e stato provato
con successo fino a 200 eventi di pileup / incrocio sulle simulazioni
per la fase ad alta luminosit\`a di ECAL.

Ho svolto l'analisi dei dati per la ricerca di materia oscura nei
canali \textit{monojet}, con un jet e energia trasversa mancante
nell'evento, con i dati del 2016, fino al limite sistematico della
stessa.

\vskip 0.5 truecm
\item \textbf{2015-- Ricercatore INFN, III livello professionale, a
  tempo indeterminato presso la sezione di Roma}
\vskip 0.2 truecm
Selezione nazionale basata su concorso pubblico per titoli ed esami.
\vskip 0.1 truecm

La mia attivit\`a scientifica riguarda prevalentemente l'esperimento
CMS presso LHC, sia negli aspetti di rivelatore (ECAL), sia
nell'analisi dei dati. Sto svolgendo misure legate al bosone W,
propedeutiche a una misura della sua massa con precisione dell'ordine
di 10$^{-4}$, che minimizzino l'uso di input teorici (e quindi le
incertezze sistematiche dominanti), quali le PDF del protone e del
modello di impulso trasverso del bosone W. Un set completo di
risultati (esclusa la massa) \`e stato pubblicato su rivista \textit{Phys.\,Rev.\,D} nel
2020. Nel periodo di shut-down dopo il Run2 mi sto anche occupando di
misure di precisione degli accoppiamenti del bosone di Higgs nel
canale in due fotoni prodotto tramite Vector Boson Fusion, e anche, in
prospettiva del Run3, di misure di produzione di multi-bosoni, in
spazi delle fasi sensibili a modificazioni dell'accoppiamento del
bosone di Higgs con il quark top.

In una frazione minore mi occupo della ricerca diretta di materia
oscura nell'esperimento CYGNO, dalla costruzione e caratterizzazione
di vari prototipi con sorgenti radioattive e fasci di elettroni. In
particolare, ho sviluppato la ricostruzione degli eventi
dell'esperimento, costituiti da immagini raccolte con una telecamera
con sensore CMOS scientifico da pi\`u di $4\cdot10^6$ pixel.  La
procedura di clustering della proiezione 2D della scintillazione
secondaria nel gas della TPC \`e un'applicazione degli algoritmi stato
dell'arte di machine learning non supervisionato. Con questo sono
stati ottenuti i primi risultati su rinculi nucleari di energia di
pochi keV prodotti da una sorgente radioattiva di Americio-Berillio,
pubblicati su rivista \textit{Meas.\,Sci.\,Techn.} nel 2020.

\end{enumerate}


\vskip 0.8 truecm
\begin{center}
\textbf{Elenco dei premi o riconoscimenti all’attivit\`a personale}
\end{center}
\vskip 0.3 truecm

\begin{enumerate}
\item \textbf{``CMS Achievement Award''} (2009). Premio attribuito dalla
  collaborazione CMS per il contributo determinante alla fase di
  commissioning di ECAL con i primi dati da LHC. Titolo: \textit{``For
  outstanding contribution in the CMS ECAL commissioning through the
  development of the ECAL high voltage system (hardware) and data
  quality monitoring (software).''}

\item \textbf{``EPS High Energy and Particle Physics Prize''} (2013),
  attribuito alle collaborazioni ATLAS e CMS dalla European Physics
  Society, \textit{``For the discovery of a Higgs boson, as predicted by
  Brout-Englert-Higgs mechanism''.}
\end{enumerate}

\end{document}
