\documentclass[11pt,twoside,a4paper]{article}
\usepackage{geometry}
\geometry{a4paper, top=2.3cm, bottom=2.3cm, left=2.3cm, right=2.3cm}
\usepackage{graphicx} 
\usepackage{marvosym}
\usepackage{xcolor}
\usepackage{xspace}
\usepackage{url}
\usepackage{amsmath}
\usepackage{cite}
\usepackage[T1]{fontenc}
\newcommand{\HRule}{\rule{\linewidth}{0.2mm}}
\thispagestyle{empty}
\setlength{\parindent}{0mm}
\setlength{\parskip}{0mm}
%%%%%%%%%%%%%%%%%%%%%%%%%%%
%% \hoffset = -2.0 cm
%% \voffset = -3.0 cm
%% \textheight = 25.0 cm
%% \marginparwidth = 1.0 cm
%% \evensidemargin = 1.0 cm
%% \textwidth =16.1 cm
%% \renewcommand{\arraystretch}{1.5}
%%%%%%%%%%%%%%%%%%%%%%%%%%%

%%% remove "References" title
\usepackage{etoolbox}
\patchcmd{\thebibliography}{\section*{\refname}}{}{}{}
%%%%%%%%%%%%%%%%%%%%%%%%%%%%%%


\begin{document}

\begin{center}
\textbf{Incarichi di responsabilit\`a o coordinamento scientifico e ruoli di servizio}
\end{center}
\vskip 0.8 truecm

\textit{Esperimento CMS}
\begin{enumerate}
\item Co-responsabile del centro di calcolo Tier-2 di Roma
  (2017--). Circa 5 persone coinvolte. Attribuito internamente dal
  gruppo CMS di Roma.

  L'attivit\`a svolta comprende la gestione del
  sistema operativo, la connettivit\`a di rete, con supporto sia
  hardware che software, per mantenere la massima affidabilit\`a del
  sito, che \`e inserito nel sistema mondiale della \textit{grid} del
  calcolo di CMS.  L'attivit\`a comprende il coordinamento
  dell'upgrade annuale hardware e software dei server di calcolo (per
  un totale di circa 2500 core) e di storage dei dati di CMS (per un
  totale di circa 2PB di spazio disco), compresi gli acquisti (per un
  investimento medio di circa 100 $kEuro$ l'anno.

\item Responsabile Unico del Procedimento (RUP) per tutte le gare e
  procedure di acquisto del gruppo di CMS Roma (2015--)

\item Coordinatore del Detector Performance Group (DPG) di ECAL
  (``Level-2 CMS coordination'', 2014--2017, circa 60 persone
  coinvolte). Attribuito dall ``Management Board'' di CMS.

  L'attivit\`a ha incluso il coordinamento del lavoro di
  ottimizzazione della presa dati del calorimetro elettromagnetico
  (trigger di livello 1, temporizzazione del rivelatore, monitoring
  della trasparenza con il laser, controllo di qualit\`a ei dati
  raccolti), della ricostruzione dell'energia dei depositi
  calorimetrici, della calibrazione della risposta di singolo
  cristallo e dei cluster di elettroni e fotoni. Il ruolo ricoperto
  implica anche il coordinamento e il collegamento tra gli aspetti di
  basso livello legati al rivelatore durante la presa dati, e le
  implicazioni sulla ricostruzione e idientificazione delle particelle
  usate nelle analisi di fisica. Una delle sinergie maggiori \`e stata
  con il gruppo per le misure sul bosone Higgs, per ottimizzare la
  sensibilit\`a al canale di decadimento $H\to\gamma\gamma$ attraverso
  la massimizzazione della risoluzione in energia. Il mio mandato \`e
  coinciso con la transizione di LHC tra 50 e 25\,ns, e quindi con
  notevole aumento del contributo del pileup \textit{out-of-time},
  cio\`e da bunch crossings adiacenti a quello dell'hard scattering.
  Ho sviluppato personalmente il nuovo algoritmo di ricostruzione
  dell'ampiezza digitizzata da ECAL (``multifit''). Come coordinatore
  del gruppo ho seguito e coordinato le validazioni delle ricadute
  sulle performance di tutte le particelle che coinvolgono ECAL (non
  solo elettroni e fotoni, ma anche jet e energia traqsversa
  mancante). Ho anche messo a punto una procedura di calibrazione
  degli input dell'algoritmo che richiede saltuariamente dei fill
  speciali di LHC con incroci di due bunches isolati. L'algoritmo \`e
  stato usato, con opportune modifiche, anche per la ricostruzione di
  ampiezza del calorimetro adronico (HCAL).

\item Coordinatore del gruppo di analisi per la ricerca e poi le
  misure del bosone di Higgs in 4 leptoni, H$\to$ZZ$\to4\ell$,
  (2013--2014, circa 20 persone coinvolte). Incarico attribuito dai
  coordinatori del gruppo Higgs di CMS.

  L'attivit\`a \`e consistita nel coordinamento dei vari approcci e
  ingredienti dell'analisi, che ha portato alla pubblicazione
  dell'articolo ``legacy'' con i dati di Run1: ottimizzazione della
  selezione e calibrazione di scala di momento di elettroni e muoni,
  fino a un impulso trasverso di 5 GeV, prima misura della massa in
  CMS, prime misure di spin-parit\`a, sia con l'approccio di un
  discriminante cinematico, sia con un fit di likelihood
  8-dimensionale, che hanno permesso i vincoli di molteplici
  accoppiamenti anomali del bosone di Higgs con i bosoni di Higgs, sia
  in decadimento, sia, per la prima volta, in produzione, attraverso
  la produzione VBF e con W e Z associati.

  \item Coordinatore della ricostruzione di elettroni e fotoni in CMS
    (2013--2014, circa 20 persone coinvolte). Attribuito dai
    coordinatori del gruppo ``EGamma'' di CMS.

    L'attivit\`a ha compreso lo sviluppo e l'integrazione della
    ricostruzione di queste particelle nella ricostruzione ``particle
    flow'' di CMS, con particolare attenzione all'attribuzione univoca
    di tracce e depositi elettromagnetici a un solo candidato
    particella nell'evento. Questa ricostruzione \`e attualmente il
    default per l'esperimento.

  \item Coordinatore del gruppo di analisi per la ricerca e poi le
    misure del bosone di Higgs in due bosoni W, H$\to$WW
    (2011--2012, circa 40 persone coinvolte). Incarico attribuito
    dai coordinatori del gruppo Higgs di CMS.

    Nel corso del mandato ho coordinato molteplici gruppi coinvolti
    nell'analisi degli stati finali completamente leptonici, che ha
    contribuito alla scoperta del bosone con massa 125 GeV come il
    canale con maggiore sensibilit\`a. Nel corso dell'analaisi sono
    stati ottimizzate le selezioni dei leptoni che sono in uso ancora
    oggi. Il gruppo ha poi esteso le prime misure alla produzione VBF
    e in associazione con W e Z, e inoltre agli stati finali non
    puramente leptonici. Lo stato finale con due leptoni ed energia
    mancante nell'evento ha costituito la base per successive analisi
    coinvolgenti il quark top (produzione $t\bar t$) e ricerche di
    SUSY con conservazione di R-parit\`a, che produce catene di
    decadimento multileptoniche e con energia mancante, dovuta alla
    particella supersimmetrica pi\`u leggera stabile.

  \item Responsabile del sistema hardware del laser di ECAL
    (2011--2014, circa 5 persone coinvolte). Responsabilit\`a di
    istituto di Caltech.

    L'attivit\`a svolta \`e stata quella delle operazioni dei laser
    blu (lunghezza d'onda 447 nm), usati per il monitoring della
    trasparenza di ECAL. Durante questo periodo ho partecipato al
    commissioning e all'operazione del laser di nuova generazione
    \textit{diod-pumped} a stato solido.

  \item Reponsabile del sistema di alta tensione per ECAL (2009-2011,
    circa 5 persone coinvolte). Responsabilit\`a di istituto di Roma.

    L'attivit\`a svolta \`e stata quella dell'installazione dei crate
    e schede dell'HV per gli APD di ECAL, della calibrazione e
    commissioning del sistema con i primi dati con i cosmici di CMS.
    Durante la prima presa dati \`e stato scoperto l'insorgere di un
    rumore di pick-up portato dai cavi ai supermoduli del
    calorimetro. Dopo aver individuato personalmente l'origine del
    rumore ho curato la campagna di modifica del grounding delle
    schede che ha permesso la rimozione di questo rumore.

  \item Responsabile del data-quality monitoring (DQM) di CMS
    (2009--2011, circa 5 persone coinvolte). Responsabilit\`a di
    istituto di Roma.

    L'attivit\`a svolta \`e stata inizialmente quella dello sviluppo
    del sistema software di DQM per il rivelatore ECAL. Dopodich\'e la
    coordinazione per il sottorivelatore ``preshower'', che ha
    utilizzato l'infrastruttura sviluppata, e infine l'integrazione
    del DQM come strumento nell'infrastruttura globale di validazione
    della ricostruzione degli eventi, comprendente anche la
    simulazione (per la parte degli oggetti di basso livello).
    
\end{enumerate}

\textit{Esperimento BaBar}
\begin{enumerate}
\item Responsabile per le misure di efficienza del flavor b-tagging
  (2006, circa 5 persone coinvolte). Responsabilit\`a attribuita dal
  coordinatore della fisica di BaBar.

  L'attivit\`a \`e consistita nel misurare le efficienze di flavor
  tagging per i mesoni $B$, usando un campione di $B$ in stati finali
  di alta purezza e completamente ricostruiti. Tali efficienze sono
  state usate dalle misure di violazione di CP dipendenti dal tempo da
  tutta la collaborazione.
\end{enumerate}

\vskip 0.8 truecm
\begin{center}
\textbf{Comitati editoriali di riviste scientifiche}
\end{center}
\vskip 0.5 truecm

Dal 2018 sono un referee di articoli scientifici per la rivista
\textit{Physics Letters B} (Elsevier, Impact Factor del 2019 4.4), per
il settore di ``Particle Physics, Nuclear Physics and Cosmology''.


\vskip 0.8 truecm
\begin{center}
\textbf{Organizzazione di congressi scientifici}
\end{center}
\vskip 0.5 truecm

Partecipazione al comitato di organizzazione locale della conferenza
internazionale ``Standard Model a LHC 2020'' - Roma, Aprile 2020. 

\end{document}
