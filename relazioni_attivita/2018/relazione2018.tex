% compile with:
% pdflatex relazione2018.tex 
% bibtex main
% bibtex all
% pdflatex relazione2018.tex 
% pdflatex relazione2018.tex 
\documentclass[a4paper,12pt,twoside]{article}
\usepackage{latexsym}
\usepackage{amssymb}
\usepackage{amsmath}
\usepackage{cite}
\usepackage{multibib}
\newcites{main}{Pubblicazioni scelte}
\newcites{all}{Lista di tutte le altre pubblicazioni}
\usepackage{filecontents}
\usepackage[english,italian]{babel}
\pagestyle{headings}
\title{dfdfd}
%\maketitle
\usepackage[dvips]{graphicx}
\usepackage{fontenc}
\usepackage{epic,rotating,epsfig}
\usepackage{times}
\usepackage{fancyheadings}
\linespread{1.2}
%%%%%%%%%%%%%%%%%%%%%%%%%%%
\hoffset = -1.0 cm
\voffset = -2.0 cm
\textheight = 24.0 cm
\marginparwidth = 1.0 cm
\evensidemargin = 1.0 cm
\textwidth =16.1 cm
%%%%%%%%%%%%%%%%%%%%%%%%%%%
\begin{document}
\fontencoding{OT1}\fontfamily{pbk}\fontseries{m}
\selectfont
\vspace*{2.20cm}

\begin{center}
  {\LARGE \bf Relazione sull'attivit\`a svolta dal 01/04/2015} \\
  \vspace*{0.2cm}
  {\large \it Emanuele Di Marco}
\end{center}

\vspace*{2.2cm}

\hspace*{-0.6cm}{ L'attivit\`a da me svolta dal 1 Aprile 2015 ad oggi
  si inserisce nella ricerca di fenomeni non previsti dal Modello
  Standard delle particelle elementari, principalmente nell'ambito
  dell'esperimento {\it CMS} al {\it Large Hadron Collider} del {\it
    CERN} di Ginevra. Sulla stessa linea, ho anche svolto un progetto
  di ricerca diretta di tali segnali, mediante un'attivit\`a di
  sviluppo di rivelatori per evidenziare rinculi nucleari di eventuali
  particelle massive di materia oscura.  Inoltre sono anche impegnato
  in una serie di misure di precisione di quantit\`a notevoli del
  Modello Standard utilizzando i dati di CMS.

  Sono stato lo sviluppatore dell'algoritmo di base di ricostruzione
  dell'energia depositata nei cristalli del calorimetro
  elettromagnetico ({\it ECAL}) di CMS, usato dall'inizio del {\it
    RunII} di CMS. L'algoritmo utilizzato durante il periodo di presa
  dati all'energia del centro di massa di LHC di 7 e 8 TeV, infatti,
  era basato su una semplice combinazione lineare dei segnali
  digitizzati dall'elettronica, ottimale per la riduzione del rumore,
  ma vulnerabile ai contributi da collisioni in {\it bunch crossings}
  (BX) diversi da quello nominale ({\it pileup ``out-of-time''}).  Il
  passaggio da collisioni distanziate di 50ns a quelle, nominali per
  LHC, di 25ns, ha reso necessario un cambio radicale. Ho disegnato,
  sviluppato e integrato l'algoritmo durante il periodo di shutdown
  dell'acceleratore.  Questo si basa su un fit che misura le ampiezze
  di 10 possibili BX intorno a quello nominale, basato su una
  linearizzazione dei minimi quadrati non negativi ({\it
    multifit}). Questo ha permesso di eliminare il deterioramento
  della risoluzione di energia dovuta al pileup in RunII. L'algoritmo
  \`e stato usato sia nella ricostruzione degli eventi {\it offline},
  che nel trigger di alto livello di CMS.  Sto scrivendo un articolo
  che documenta questa tecnica innovativa, che include le performances
  ottenute nei dati raccolti dal 2015 ad oggi.

  Oltre alla ricostruzione di energia, ho anche sviluppato algoritmi
  atti a sostenere una luminosit\`a istantanea di 1.5 $\times$
  10$^{34}$cm$^{-2}$s$^{-1}$, e mi sono occupato personalmente della
  calibrazione della risposta in energia dei cristalli mediante la
  risonanza $\pi^0\to\gamma\gamma$.

  Dal Gennaio 2014 a Settembre 2017 sono stato il coordinatore del
  gruppo dedicato alle performances di ECAL. Questo gruppo, in cui
  lavorano attivamente circa 80 persone, si occupa dell'ottimizzazione
  delle capacit\`a del rivelatore di misurare gli oggetti
  elettromagnetici, quali elettroni e fotoni, in un intervallo di
  energia esteso da centinaia di MeV al
  TeV~\citemain{Khachatryan:2015hwa,Khachatryan:2015iwa}. Ho
  coordinato gli studi necessari alla presa dati del rivelatore,
  ricostruzione di osservabili di basso livello ({\it hit, cluster}),
  calibrazioni di singolo hit e risposta in energia dei cluster
  elettromagnetici che sono stati usati nella prima misura in RunII
  delle propriet\`a del bosone di Higgs nei decadimenti in due fotoni
  (prima osservazione con significativit\`a $>5\sigma$ utilizzando
  solo questo canale), con una determinazione della sua massa
  $m_H=125.4$ GeV $\pm0.3$ GeV.  I risultati che utilizzano il dataset
  del 2015 e 2016 sono in corso di pubblicazione.

  Questi risultati sono in accordo con le misure fatte durante il Run
  da ATLAS e CMS, i cui risultati, compatibili tra loro, sono stati
  combinati. In questa combinazione ho partecipato in quanto sono
  stato autore principale della ricerca, nei dati del RunI, nei canali
  $WW$ e $ZZ$ nei decadimente completamente
  leptonici~\citemain{Khachatryan:2014jba,Aad:2015zhl}, in molti
  aspetti (sezione d'urto, massa, spin/parit\`a
  ~\citemain{Khachatryan:2014kca}).

  La calibrazione della scala ad alte
  energie \`e stata utilizzata nella ricerca di risonanze ad alta
  massa in di-fotoni~\citemain{Khachatryan:2016hje,Khachatryan:2016yec}.

  Oltre che agli aspetti legati al rivelatore, mi sono occupato
  dell'analisi dei dati raccolti in RunII. Nel 2015 e 2016 nell'ambito
  di CMS ho contribuito alla ricerca di una possibile produzione a LHC
  di particelle massive responsabili della presenza di materia oscura
  ({\it WIMP}), la cui esistenza \`e provata da osservazioni
  astrofisiche, come la velocit\`a di rotazione delle galassie. Ho
  contribuito alla ricerca misurando la {\it rate} di produzione di
  jet e energia mancante trasversa negli eventi, in cui l'eventuale
  produzione di WIMP (non interagenti con il rivelatore) \`e associata
  a jet di alto $P_T$ che permettono di individuare l'evento.
  L'assenza di segnale rivelato, rispetto ai fondi attesi da processi
  del Modello Standard, in un campione di dati di 35.9$fb^{-1}$ \`e
  stata utlizzata per porre dei limiti molto stringenti sulla
  produzione di materia oscura in modelli semplificati in cui il
  mediatore ha spin 0 (con masse fino a 400 GeV) o 1 (con masse fino a
  1.8 TeV)~\citemain{Sirunyan:2017hci}.  Il risultato pone anche un
  vincolo sul {\it branching ratio} di decadimento in particelle
  invisibili del bosone di Higgs di tipo Standard Model
  $<53$\%~\citemain{Khachatryan:2016whc}.
  
  Dati i limiti stringenti posti da queste ed altre misure di LHC su
  segnali di nuova fisica, dall'inizio del 2017 mi occupo anche di
  misure di fisica elettrodebole. In particolare, una delle misure
  pi\`u ambiziose in questo campo \`e quella della massa del bosone
  $W$. Questa misura \`e importante nell'ambito del fit elettrodebole,
  in quanto, insieme alla massa del quark top, pone un vincolo alla
  massa del bosone di Higgs. Il livello di precisione richiesto \`e
  estremo (circa 10 MeV), che implica la necessit\`a della conoscenza
  dell'impulso dei leptoni (muoni o elettroni), dell'ordine di
  $\sigma_{P_T}/P_T \approx 10^{-4}$. Insieme ad altri membri del
  gruppo CMS di Roma abbiamo iniziato l'analisi, includendo anche il
  canale di decadimento in elettrone, in cui sfrutto l'asperienza
  generale nell'analisi di dati e quella particolare sulla
  calibrazione della scala di energia degli elettroni.  Questa misura
  richiede anche un programma pi\`u esteso di misure, che sono
  propedeutiche alla misura di $m_W$, quali la sezione d'urto
  differenziale del bosone $Z$, in quanto permette di vincolare le
  incertezze teoriche legate alla produzione del bosone $W$. Per far
  questo, ho supervisionato e sto supervisionando anche studenti di
  laurea e dottorato dei gruppi di CMS di Roma e Trieste con cui
  collaboro.

  Nell'ambito delle misure elettrodeboli, \`e stata finalizzata la
  pubblicazione della misura della sezione d'urto di coppie di bosoni
  $W$ a 8 TeV, che \`e stato un passo propedeutico alla ricerca del
  bosone di Higgs nel canale $WW$ in RunI~\citemain{Khachatryan:2015sga}.
  
  Durante gli anni dal 2015 al 2017 ho partecipato a un progetto di
  ricerca e sviluppo per rivelatori di segnali dovuti ai rinculi di
  possibili WIMP con i nuclei della materia ordinaria (nell'ambito
  della ricerca diretta di segnali di materia oscura). Questo
  approccio permette una sensibilit\`a complementare alla ricerca di
  CMS. Quelli studiati sono rivelatori a gas del tipo TPC ({\it Time
    Projection Chambers}), che sfruttano la bassa densit\`a del mezzo
  per permettere al rinculo nucleare di lasciare una traccia
  sufficientemente lunga da essere rivelata. La piccola probabilit\`a
  di interazione per\`o obbliga ad avere volumi di gas di decine o
  centinaia di $m^3$. Lo sviluppo cui ho contribuito \`e stato quello
  delle possibili rivelazione di tali segnali, mediante lettura
  ottica.  Questa si basa su fotocamere ad alta precisione usate per
  il tracciamento, eventualmente accoppiate con fotorivelatori molto
  veloci che potrebbero essere usati per la ``fiducializzazione'' del
  volume (tecnica per rigettare fondi da radioattivit\`a naturale
  sulle pareti del rivelatore). Ho contribuito al setup sperimentale
  di tre test alla BTF di Frascati su fascio di elettroni, e alla
  prima analisi dei dati.
  

}

\bibliographystylemain{unsrt}
\bibliographymain{relazione2018}

\citeall{*}
\bibliographystyleall{unsrt}
\bibliographyall{allpubs}


\end{document}




