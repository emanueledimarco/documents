% compile with:
% makeTex relazione2022
\documentclass[a4paper,12pt,twoside]{article}
\usepackage{latexsym}
\usepackage{amssymb}
\usepackage{amsmath}
\usepackage{cite}
\usepackage{multibib}
\newcites{main}{Pubblicazioni scelte}
\newcites{all}{Lista di tutte le altre pubblicazioni}
\usepackage{filecontents}
\usepackage[english,italian]{babel}
\pagestyle{headings}
\title{dfdfd}
%\maketitle
\usepackage[dvips]{graphicx}
\usepackage{fontenc}
\usepackage{epic,rotating,epsfig}
\usepackage{times}
\usepackage{fancyheadings}
\linespread{1.2}
%%%%%%%%%%%%%%%%%%%%%%%%%%%
\hoffset = -1.0 cm
\voffset = -2.0 cm
\textheight = 24.0 cm
\marginparwidth = 1.0 cm
\evensidemargin = 1.0 cm
\textwidth =16.1 cm
%%%%%%%%%%%%%%%%%%%%%%%%%%%
\begin{document}
\fontencoding{OT1}\fontfamily{pbk}\fontseries{m}
\selectfont
\vspace*{2.20cm}

\begin{center}
  {\LARGE \bf Relazione sull'attivit\`a svolta dal 01/04/2018} \\
  \vspace*{0.2cm}
  {\large \it Emanuele Di Marco}
\end{center}

\vspace*{2.2cm}

\hspace*{-0.6cm}{ L'attivit\`a da me svolta dal 1 Aprile 2018 ad oggi
  si inserisce nella ricerca di fenomeni non previsti dal Modello
  Standard delle particelle elementari, nonch\'e di misure di
  parametri fondamentali dello stesso, soprattutto nell'ambito
  dell'esperimento {\it CMS} al {\it Large Hadron Collider} del {\it
    CERN} di Ginevra. In contemporanea ho sviluppato il lavoro
  iniziato dalla data della mia assunzione del 2015 su un progetto di
  ricerca diretta di segnali di materia oscura, dallo sviluppo di
  rivelatori che sfrutti rinculi nucleari di eventuali particelle
  massive di materia oscura, portandolo dalla fase di studio di
  fattibilit\`a a un primo prototipo in grado di prendere dati ai {\it
    Laboratori Nazionali del Gran Sasso}, cosiddetto esperimento {\it
    CYGNO}.

  Nel periodo dal 2018 ad oggi ho continuato la mia attivit\`a di
  comprensione, miglioramento delle prestazioni del calorimetro
  elettromagnetico di CMS, {\it ECAL}, attraverso la calibrazione
  delle {\it pulse shape} dei fotorivelatori dei cristalli, usate come
  input dell'algoritmo di ricostruzione dell'energia che ho sviluppato
  negli anni precedenti e usato per tutto il Run-2 di LHC. Essa ha
  permesso di far rimanere inalterata la risoluzione energetica di
  ECAL rispetto agli effetti causati dal pileup, fino a 60 interazioni
  per incrocio dei fasci, valore massimo durante il Run2.  Il
  funzionamento dell'algoritmo \`e stato dimostrato su simulazioni
  fino a una media di 200 interazioni per incrocio dei fasci, e si
  prevede di usarlo per la fase di alta luminosit\`a di LHC: in questo
  caso le prestazioni del multifit migliorano ancora, grazie alla
  maggiore frequenza di campionamento dell'impulso. L'algoritmo e le
  sue performance ottenute sui dati di Run2 Nello stesso ambito, mi
  sono occupato personalmente della calibrazione della risposta in
  energia dei cristalli mediante la risonanza $\pi^0\to\gamma\gamma$,
  e questi input sono stati fndamentali nelle analisi del bosone di
  Higgs nei decadimenti in due fotoni, ma anche per tutte le analisi
  di CMS che, per i risultati aggiornati all'intero data set di Run-2
  (${\cal L}_{\rm int}=137\,{\rm fb}^{-1}$), hanno utilizzato la
  ricostruzione cosiddetta {\it Ultra-Legacy}.

  Sono stato il chair di tutte le analisi che hanno studiato i
  decadimenti $H\to\gamma\gamma$, dalla misura della massa del bosone
  H, alle misure delle sezioni d'urto differenziali, alla prima misura
  della produzione associata $t\bar{t}$H. Questi risultati sono in
  accordo con le misure fatte durante il Run-1 da ATLAS e CMS, i cui
  risultati, compatibili tra loro, sono stati combinati. In particolare
  gli studi sulla scala di energia di ECAL durante i tre anni 2016-2018 sono
  stati un ingrediente fondamentale di tutte queste misure. 

  Sono membro dal 2019 del {\it Publication Committee} del gruppo del
  bosone di Higgs di CMS e, da Settembre del 2021, sono chair dello
  stesso comitato. Nel corso di questo periodo ho coordinato le
  pubblicazioni di tutti gli articoli (circa 40) sul bosone di Higgs,
  tra cui anche l'articolo pubblicato sulla rivista Nature con il
  sommario di tutte le misure aggiornate al campione di dati di Run-2
  in occasione dei 10 anni dalla scoperta dello stesso. Sono
  attualmente anche l'autore principale, della misura degli
  accoppiamenti del bosone di Higgs ai bosoni vettori massivi W e Z,
  usando produzione VBF e produzione associata con W e Z: sono stato
  il relatore di 3 tesi di laurea su questi argomenti, nonch\'e di
  numerose dissertazioni di laurea triennale alla Sapienza.

  Dal 2017 al 2020 sono stato l'autore principale delle misure
  collegate alla determinazione della massa del bosone W con
  l'esperimento CMS. In particolare ho messo a punto la misura della
  sezione d'urto di produzione del bosone W, multi-differenziale
  nell'elicit\`a, rapidit\`a del W, e inoltre doppio-differenziale
  rispetto al momento trasverso e pseudo-rapidit\`a del leptone di
  decadimento, combinando gli stati finali con elettrone e muone. La
  misura \`e di rilevanza nel programma di misura della massa del
  bosone W~[c17] che ha l'obiettivo di raggiungere una precisione
  $\delta m_W/m_W\approx 10^{-4}$, poich\'e permette di vincolare
  in-situ le PDF del protone nell'intervallo di momento rilevante. La
  tecnica di unfolding sperimentale \`e all'avanguardia, avendo
  sviluppato a tal scopo un algoritmo di minimizzazione della
  likelihood alternativo a \textsc{MINUIT}, basato su
  \textsc{TENSORFLOW}, che permette di trattare likelihood molto
  complesse, con cuspidi locali e un numero di parametri misurati
  simultaneamente superiore al migliaio. I risultati dell'analisi e la
  descrizione degli strumenti innovativi sono stati pubblicati sulla
  rivista \textit{Phys.\,Rev.\,D} nel 2020. La stessa tecnica di
  misura e strumenti matematici si stanno usando per la prima misura
  della massa del W in CMS, basandosi solamente sulla cinematica del
  leptone, allo scopo di ridurre al minimo gli input teorici esterni,
  come le PDF e il modello dello spettro in $p_T$ del bosone W.  Ci\`o
  permetter\`a di minimizzare le sistematiche dominanti di tale misura
  ad un collisore pp, quale LHC.


  
  Dati i limiti stringenti posti da queste ed altre misure di LHC su
  segnali di nuova fisica, dall'inizio del 2017 mi occupo anche di
  misure di fisica elettrodebole. In particolare, una delle misure
  pi\`u ambiziose in questo campo \`e quella della massa del bosone
  $W$. Questa misura \`e importante nell'ambito del fit elettrodebole,
  in quanto, insieme alla massa del quark top, pone un vincolo alla
  massa del bosone di Higgs. Il livello di precisione richiesto \`e
  estremo (circa 10 MeV), che implica la necessit\`a della conoscenza
  dell'impulso dei leptoni (muoni o elettroni), dell'ordine di
  $\sigma_{P_T}/P_T \approx 10^{-4}$. Insieme ad altri membri del
  gruppo CMS di Roma abbiamo iniziato l'analisi, includendo anche il
  canale di decadimento in elettrone, in cui sfrutto l'asperienza
  generale nell'analisi di dati e quella particolare sulla
  calibrazione della scala di energia degli elettroni.  Questa misura
  richiede anche un programma pi\`u esteso di misure, che sono
  propedeutiche alla misura di $m_W$, quali la sezione d'urto
  differenziale del bosone $Z$, in quanto permette di vincolare le
  incertezze teoriche legate alla produzione del bosone $W$. Per far
  questo, ho supervisionato e sto supervisionando anche studenti di
  laurea e dottorato dei gruppi di CMS di Roma e Trieste con cui
  collaboro.

  Nell'ambito delle misure elettrodeboli, \`e stata finalizzata la
  pubblicazione della misura della sezione d'urto di coppie di bosoni
  $W$ a 8 TeV, che \`e stato un passo propedeutico alla ricerca del
  bosone di Higgs nel canale $WW$ in RunI~\citemain{Khachatryan:2015sga}.
  
  Durante gli anni dal 2015 al 2017 ho partecipato a un progetto di
  ricerca e sviluppo per rivelatori di segnali dovuti ai rinculi di
  possibili WIMP con i nuclei della materia ordinaria (nell'ambito
  della ricerca diretta di segnali di materia oscura). Questo
  approccio permette una sensibilit\`a complementare alla ricerca di
  CMS. Quelli studiati sono rivelatori a gas del tipo TPC ({\it Time
    Projection Chambers}), che sfruttano la bassa densit\`a del mezzo
  per permettere al rinculo nucleare di lasciare una traccia
  sufficientemente lunga da essere rivelata. La piccola probabilit\`a
  di interazione per\`o obbliga ad avere volumi di gas di decine o
  centinaia di $m^3$. Lo sviluppo cui ho contribuito \`e stato quello
  delle possibili rivelazione di tali segnali, mediante lettura
  ottica.  Questa si basa su fotocamere ad alta precisione usate per
  il tracciamento, eventualmente accoppiate con fotorivelatori molto
  veloci che potrebbero essere usati per la ``fiducializzazione'' del
  volume (tecnica per rigettare fondi da radioattivit\`a naturale
  sulle pareti del rivelatore). Ho contribuito al setup sperimentale
  di tre test alla BTF di Frascati su fascio di elettroni, e alla
  prima analisi dei dati.
  

}

\bibliographystylemain{unsrt}
\bibliographymain{relazione2018}

\citeall{*}
\bibliographystyleall{unsrt}
\bibliographyall{allpubs}


\end{document}




