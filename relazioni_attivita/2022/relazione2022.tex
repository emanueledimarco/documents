% compile with:
% makeTex relazione2022
\documentclass[a4paper,12pt,twoside]{article}
\usepackage{latexsym}
\usepackage{amssymb}
\usepackage{amsmath}
\usepackage{cite}
\usepackage{multibib}
\newcites{main}{10 pubblicazioni scelte dal 2018 al 2022}
\newcites{all}{Lista di tutte le pubblicazioni dal 2018 al 2022}
\usepackage{filecontents}
\usepackage[english,italian]{babel}
\pagestyle{headings}
\title{dfdfd}
%\maketitle
\usepackage[dvips]{graphicx}
\usepackage{eurosym}
\usepackage{fontenc}
\usepackage{epic,rotating,epsfig}
\usepackage{times}
\usepackage{fancyheadings}
\linespread{1.2}
%%%%%%%%%%%%%%%%%%%%%%%%%%%
\hoffset = -1.0 cm
\voffset = -2.0 cm
\textheight = 24.0 cm
\marginparwidth = 1.0 cm
\evensidemargin = 1.0 cm
\textwidth =16.1 cm
%%%%%%%%%%%%%%%%%%%%%%%%%%%
\begin{document}
\fontencoding{OT1}\fontfamily{pbk}\fontseries{m}
\selectfont
\vspace*{2.20cm}

\begin{center}
  {\LARGE \bf Relazione sull'attivit\`a svolta dal 2018 al 2022} \\
  \vspace*{0.2cm}
  {\large \it Emanuele Di Marco}
\end{center}

\vspace*{2.2cm}

\hspace*{-0.6cm}{ L'attivit\`a da me svolta dal 1 Aprile 2018 ad oggi
  si inserisce nella ricerca di fenomeni non previsti dal Modello
  Standard delle particelle elementari, nonch\'e di misure di
  parametri fondamentali dello stesso, soprattutto nell'ambito
  dell'esperimento {\it CMS} al {\it Large Hadron Collider} (LHC) del
  {\it CERN} di Ginevra. Allo stesso tempo ho sviluppato il lavoro
  relativo all'esperimento {\it CYGNO}, per la ricerca di segnali di
  materia oscura, portandolo dallo stadio di studio di fattibilit\`a
  all'installazione di un primo prototipo in grado di prendere dati ai
  {\it Laboratori Nazionali del Gran Sasso}.

  Nel periodo dal 2018 ad oggi ho proseguito la mia attivit\`a di
  comprensione e miglioramento delle prestazioni del calorimetro
  elettromagnetico di CMS ({\it ECAL}), attraverso la calibrazione
  delle {\it pulse shape} dei fotorivelatori dei cristalli, che sono
  usate come input dell'algoritmo di ricostruzione dell'energia da me
  proposto e sviluppato negli anni precedenti, e che \`e stato usato
  per tutto il Run-2 di LHC. Essa ha permesso di far rimanere
  inalterata la risoluzione energetica di ECAL rispetto agli effetti
  causati dal pileup, fino a 60 interazioni per incrocio dei fasci,
  valore massimo durante il Run-2.  Il funzionamento dell'algoritmo
  \`e anche stato dimostrato su simulazioni fino a una media di 200
  interazioni per incrocio dei fasci, e si prevede di usarlo per la
  fase di alta luminosit\`a di LHC. Ho scritto un articolo pubblicato
  su \textit{JINST} su questo~\citemain{sp_CMS:2020xlg}. Sempre
  nell'ambito delle prestazioni di ECAL, mi sono occupato della
  calibrazione della risposta in energia dei cristalli mediante la
  risonanza $\pi^0\to\gamma\gamma$, e questa misura \`e stata
  fondamentale nelle analisi dati riguardanti i decadimenti del bosone
  di Higgs in due fotoni~\citemain{sp_CMS:2020xrn}. In generale queste
  calibrazioni sono state usate per tutte le analisi di CMS che hanno
  usato l'intero data set di Run-2 (${\cal L}_{\rm int}=137\,{\rm
    fb}^{-1}$). Inoltre, a partire dal 2021, mi sto occupando del
  progetto di upgrade del sistema di HV degli APD di ECAL, per quanto
  riguarda il rifacimento dei cavi e di tutte le schede che contengono
  i canali HV (lato tecnico e setup dell'eventuale gara per la
  fornitura), per un graduale rimpiazzo degli stessi entro l'inizio
  della fase-2 di LHC (HL-LHC).

  Sono stato il chair del comitato di review interno di CMS per tutte
  le analisi che hanno studiato i decadimenti $H\to\gamma\gamma$,
  dalla misura della massa del bosone H, alle misure delle sezioni
  d'urto differenziali~\citemain{sp_CMS:2021kom}, alla prima misura
  della produzione associata $t\bar{t}$H.  Dal 2021 mi sto ocupando
  personalmente della misura degli accoppiamenti del bosone di Higgs
  ai bosoni vettori massivi, usando produzione VBF e produzione
  associata con W e Z e il decadimento in due fotoni: sono stato il
  relatore di 3 tesi di laurea magistrale alla Sapienza su questi
  argomenti. I dati di questo decadimento dovrebbero fornire una
  sensibilit\`a simile a quella del canale di decadimento H$\to$ZZ$\to
  4\ell$, al quale ho precedentemente
  partecipato~\citemain{sp_CMS:2021nnc}.
  
  Sono membro dal 2019 del {\it Publication Committee} del gruppo del
  bosone di Higgs di CMS e, dal Settembre 2021, sono chair dello
  stesso comitato. Nel corso di questo periodo ho coordinato le
  pubblicazioni di tutti gli articoli (circa 40) sul bosone di Higgs,
  tra cui la prima misura della larghezza intrinseca di questa
  particella~\citemain{sp_CMS:2022ley} e l'articolo pubblicato sulla
  rivista Nature con il sommario di tutte le misure aggiornate al
  campione di dati di Run-2 in occasione dei 10 anni dalla scoperta
  dello stesso~\citemain{sp_CMS:2022dwd}.

  Dal 2017 al 2020 sono stato l'autore principale delle misure
  collegate alla determinazione della massa del bosone W con
  l'esperimento CMS. In particolare ho messo a punto la misura della
  sezione d'urto di produzione del bosone W, multi-differenziale
  nell'elicit\`a, rapidit\`a del W, e inoltre doppio-differenziale
  rispetto al momento trasverso e pseudo-rapidit\`a del leptone di
  decadimento, combinando gli stati finali con elettrone e muone. La
  misura \`e di rilevanza nel programma di misura della massa del
  bosone W, che ha l'obiettivo di raggiungere una precisione $\delta  m_W/m_W\approx 10^{-4}$,
  poich\'e permette di vincolare in-situ le
  PDF del protone nell'intervallo di momento rilevante. La tecnica di
  unfolding sperimentale \`e all'avanguardia, avendo sviluppato a tal
  scopo un algoritmo di minimizzazione della likelihood alternativo a
  \textsc{MINUIT}, basato su \textsc{TENSORFLOW}, che permette di
  trattare likelihood molto complesse, con cuspidi locali e un numero
  di parametri misurati simultaneamente superiore al migliaio. I
  risultati dell'analisi e la descrizione degli strumenti innovativi
  sono stati pubblicati sulla rivista \textit{Phys.\,Rev.\,D} nel
  2020~\citemain{sp_CMS:2020cph}. La stessa tecnica di misura e strumenti
  matematici si stanno usando per la prima misura della massa del W in
  CMS, basandosi solamente sulla cinematica del leptone, allo scopo di
  ridurre al minimo gli input teorici esterni, come le PDF e il
  modello dello spettro in $p_T$ del bosone W.  Ci\`o permetter\`a di
  minimizzare le sistematiche dominanti di tale misura ad un collisore
  pp, quale LHC.

  Durante il periodo in considerazione, sono stato (e sono ancora)
  co-responsabile del centro di calcolo (\textit{Tier-2}) per i dati
  di CMS di Roma, un sistema condiviso con l'esperimento ATLAS. Esso
  \`e costituito da dieci rack che contengono qualche centinaio di
  server di elaborazione dati (per un totale di circa 2500 core) e
  archiviazione dei dati (per un totale di circa 2\,PB). Il centro di
  calcolo permette l'eleborazione di circa il 20\% dei dati raccolti
  dal LHC ogni anno, ed \`e connesso al sistema di \textit{grid}
  mondiale. In questi anni mi sono occupato della pianificazione degli
  upgrade e degli acquisti per il Tier-2 di CMS, per un investimento
  medio di circa 100\,k\euro{} all'anno.

  Nel suddetto periodo ho continuato la partecipazione allo sviluppo
  dell'esperimento CYGNO~\citemain{sp_Pinci:2019ztr}, un rivelatore
  per la ricerca diretta di materia oscura di tipo \textit{Weak
    Interacting Massive Particle} (WIMP), che si manifesti nei rinculi
  delle ipotetiche particelle sui nuclei di atomi di gas, con energia
  cinetica di pochi keV. Nel periodo 2015-2018 mi sono occupato della
  caratterizzazione di diversi prototipi del rivelatore, costituito da
  una TPC a gas con rivelatori GEM a lettura ottica mediante una
  telecamera con sensore CMOS sensibile al singolo fotone. Negli anni
  successivi, fino ad oggi, ho sviluppato la ricostruzione degli
  eventi, rappresentati dalle immagini registrate dal sensore con
  pi\`u di 4$\times$10$^6$ pixel.  Essa \`e principalmente un algoritmo
  di clustering basato su tecniche di machine learning non
  supervisionato, in grado di ricostruire efficientemente sia i
  pattern semplici di depositi di energia di raggi X prodotti da
  sorgenti radioattive, quali il
  $^{55}$Fe~\citemain{sp_Costa:2019tnu}, ma anche tracce pi\`u lunghe
  e complesse, prodotte da raggi cosmici o dalla radioattivit\`a
  naturale. Ho analizzato i dati raccolti dal prototipo {\it LEMON} ai
  \textit{Laboratori Nazionali di Frascati} con varie sorgenti radioattive, e
  con i raggi cosmici, in particolare con rinculi nucleari di energie
  cinetiche nell'intervallo di interesse per candidati WIMP, prodotti
  da una sorgente di $^{241}$Am contenuta in una capsula di berillio,
  e scritto e pubblicato un articolo su
  questo~\citemain{sp_Baracchini:2020nut}.  Nel 2022 il prototipo {\it
    LIME} \`e stato portato ai Laboratori Nazionali del Gran Sasso, e
  sono parte attiva nel commissioning del rivelatore grazie
  all'analisi dei primissimi dati {\it underground}.  Dal 2018 sono
  anche il coordinatore del gruppo di analisi dati dell'esperimento
  CYGNO, e mi occupo del codice e dell'infrastruttura per la
  ricostruzione ``prompt'' dei dataset raccolti dall'esperimento,
  usata da tutti i membri della collaborazione internazionale. Dal
  2019 al 2021 sono stato anche membro del \textit{Publication
    Committee} di CYGNO.

  Durante il periodo in considerazione mi sono occupato anche di
  didattica relativa al mio campo di interesse, come esercitatore per
  il corso di Fisica Nucleare e Subnucleare 1 (dall'a.a. 2020-2021 ad
  oggi), come tutor di esperienze di laboratorio per il corso di
  Laboratorio di Fisica Nucleare 2 (dall'a.a. 2019-2020 ad oggi), come
  relatore di 3 tesi magistrali, 1 tesi di dottorato e numerose
  dissertazioni triennali.

  Inoltre in questo periodo sono stato referee articoli scientifici
  per la rivista \textit{Physics Letters B} (nel settore di
  ``Particle Physics, Nuclear Physics and Cosmology''), per la rivista
  \textit{JHEP}, per la rivista ``Machine Learning: Science and
  Technology'', ed editor della sezione ``Upgrades in High Energy
  Physics Experiments'' della rivista \textit{Symmetry} di MDPI.
  
}

\bibliographystylemain{unsrt}
\bibliographymain{mainpub}

\citeall{*}
\bibliographystyleall{unsrt}
\bibliographyall{INSPIRE-CiteAll-2018-2022}


\end{document}




