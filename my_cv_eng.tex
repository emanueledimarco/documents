\documentclass[11pt,twoside,a4paper]{article}
%\usepackage[english,italian]{babel}
\usepackage{latexsym}
\usepackage{times}
\newcommand{\HRule}{\rule{\linewidth}{0.2mm}}
\usepackage{fancyheadings}
\thispagestyle{empty}
\setlength{\parindent}{0mm}
\setlength{\parskip}{0mm}
\usepackage[dvips]{graphicx}
\usepackage{fontenc}
\usepackage{eucal}
\usepackage{amsfonts}
\usepackage{epic,rotating,epsfig}
\usepackage{verbatim}
\usepackage{syntonly}
\usepackage{amssymb}
\usepackage{amsmath}
\usepackage{hyperref}
\usepackage{cite}
\pagestyle{empty}
%%%%%%%%%%%%%%%%%%%%%%%%%%%
\hoffset = -2.0 cm
\voffset = -3.0 cm
\textheight = 25.0 cm
\marginparwidth = 1.0 cm
\evensidemargin = 1.0 cm
\textwidth =16.1 cm
\renewcommand{\arraystretch}{1.5}
%%%%%%%%%%%%%%%%%%%%%%%%%%%


\def\bibname{ } 
\def\nextref{\global\advance\refno by 1 \number\refno\relax}
%
\begin{document}


\large
\centerline{\bf Curriculum Vit\ae{} of Emanuele Di Marco}

\vskip 0.5 truecm

\begin{itemize} 
\item  Name:  Di Marco
\item  First Name: Emanuele
\item  Born: February, $7^{th}$, 1979, Roma (Italy)
\item  Citizenship: Italian
\item  Professional address (Roma): Sapienza, Univ. Roma, Piazzale A. Moro, 2, 00185 Roma (Italy) 
%\item  Private address: via Nostra Signora di Lourdes, 126, 00167 Roma (Italy)
\item  Phone: +39 06 4991 4388
\item  e-mail: emanuele.dimarco@roma1.infn.it
\item  Languages: Mother tongue Italian. Speaks, reads, and writes fluently English and French. 
\end{itemize}

\vskip 1.0 truecm

\begin{center}
{\bf{Academic Positions}}
\end{center}
\begin{tabular}{llr}
Permanent reasercher (``III Livello'') & INFN Roma & 2015 -- present \\
``Marie Curie'' Fellow & CERN, Geneva (CH) & 2014 -- 2017 \\
``Tolman'' Prize Fellow & Caltech, Pasadena (USA) & 2011 -- 2014 \\
Postdoctoral research associate & La Sapienza Universit\`a di Roma & 2009 -- 2011 \\
INFN fellowship & CERN, Geneva (CH) & 2008 -- 2009 \\
Postdoctoral research associate & La Sapienza Universit\`a di Roma & 2007 -- 2009 \\
\end{tabular}


\vskip 1.0 truecm


\begin{center}
{\bf{Education}}
\end{center}
\begin{tabular}{lp{11cm}r}
  Ph.D. in Physics & La Sapienza Universit\`a di Roma 
  \newline Thesis: {\it Measurements of $CP$ violating asymmetries in
    charmless three-body $B$ decays with the $BaBar$ experiment}.
  Supervisor Prof. F.~Ferroni. & 2003 -- 2007 \\
  ``Laurea'' in Physics & La Sapienza Universit\`a di Roma 
  \newline Thesis: {\it Study of the decays $B^0 \rightarrow \phi K^0$ with the $BaBar$ experiment},
  (110/110). Supervisor Prof. F. Ferroni. & 1998 -- 2002 \\
  Diploma & Liceo Scientifico Statale Talete, Roma (60/60) & 1993 -- 1998 \\
\end{tabular}


\vskip 1.0 truecm


\begin{center}
{\bf{Physics Experiments}}
\end{center}
\begin{tabular}{lp{11cm}r}
CMS & Higgs boson physics, $W$ and $Z$ physics, dark matter searches, \newline ECAL software and hardware & 2007 -- present \\
CYGNUS & TPC R\&D, test-beams data analysis &  2016 -- present \\
BaBar & CP violation in $b \to s$ decays, RPC commissioner, IFR upgrade & 2002 -- 2007 \\
\end{tabular}

\vskip 1.0 truecm

\begin{center}
{\bf{Academic Titles}}
\end{center}
\begin{tabular}{lp{9cm}r}
ASN & Seconda Fascia
settore conc. 02/A1 \newline Fisica Sperimentale delle interazioni fondamentali & 2012 -- present \\
Idoneit\`a INFN & Selection for permanent researcher (``III Livello'') \newline experimental subnuclear physics & 2010 \\
Idoneit\`a INFN & Selection for temporary researcher (``III Livello'') \newline experimental physics & 2009 \\
\end{tabular}

\vskip 1.0 truecm

\begin{center}
{ \bf Teaching and tutoring Experience }
\end{center}
 \begin{itemize}  
 \item 2016--present: co-advisor of one Ph.D. thesis of CMS Roma group (M. Cipriani)
 \item in 2015, co-advisor of one master thesis of CMS Roma group (M. Cipriani)
 \item in 2016--2017, advisor of two master thesis of Trieste CMS group (A. Da Rold, J. Magro)
 \item in 2011--2014, advisor of 4 bachelor thesis of the Caltech group
 \item in 2011, advisor of one master thesis of Trieste CMS group (A. Schizzi)
 \item in 2005, advisor of one master thesis of Torino BaBar group (M. Pelliccioni)
 \item academic year 2004-2005: teaching assistant in Laboratory
   of Informatics for Physicists for undergraduate students at
   Universit\`a di Roma ``La Sapienza''.
\end{itemize}

\newpage

\begin{center}
{\bf{Project leaderships}}
\end{center}

\begin{tabular}{llr}
CMS & Rome computing center co-coord. (5 people) & 2017 -- present \\
    & ECAL detector performance group (80 people) & 2014 -- 2017 \\
    & $H \to ZZ\to4\ell$ analysis coordinator (20 people) & 2013 -- 2104 \\
    & ECAL institution board memeber (Caltech) & 2013 -- 2014 \\
    & Electron and photon reconstruction sub-group (20 people) & 2013 -- 2014 \\
    & $H \to WW$ analysis group (40 people) & 2011 -- 2012 \\
    & ECAL laser monitoring system (hardware) & 2011 -- 2014 \\
    & ECAL high voltage system & 2009 -- 2011 \\
    & ECAL data quality monitoring (5 people) & 2009 -- 2011 \\
BaBar & Flavor b-tagging efficiencies responsible & 2006 \\
      & Prompt reconstruction farm in Padova (5 people) & 2006 \\
      & Tracking reconstruction task force & 2005 \\
\end{tabular}

\vskip 1.0 truecm

\begin{center}
{\bf{Books}}
\end{center}

\begin{itemize}
\item Co-editor of the review book {\it Discovery of the
  Higgs Boson}, documenting the studies done at high energy colliders
  (LEP, Tevatron and LHC) for the search of the Higgs boson. Published
  by World Scientific in 2016 (ISBN: 978-981-4425-44-5).
\end{itemize}


\vskip 1.0 truecm


\begin{center}
  {\bf{Awards}}
\end{center}

\begin{itemize}
\item 2009: ``CMS Achievement Award''. Award for outstanding
  contribution in the $CMS$ $ECAL$ commissioning through the
  development of the $ECAL$ high voltage system (hardware) and data
  quality monitoring (software).
\end{itemize}

\clearpage
\newpage 

\begin{center}
{\bf{Research activity}}
\end{center}
My research activity focused on the experimental high energy physics.

From 2002 to 2007 it focused on the study of CP violation in the $B$
meson weak decays and the search of indirect signs of physics beyond
the Standard Model in the flavor sector through charmless $b \to s$
decays. Measurement of the time-dependent CP violation of the $B$
meson in three kaon decays, first in the quasi-two-body
approximation~\cite{Aubert:2005ja,conf_aps,conf_win05}, then including
interference effects through the Dalitz
plot~\cite{Aubert:2007sd,Aubert:2007mj,talk_nagoya,DiMarco:2006wg,talk_ichep}.
A new spin-zero resonance decaying in two charged kaons with mass
around 1.55 GeV/c$^2$, the $X_0(1550)$, have been found as
contributing to the Dalitz plot. Usage of ad hoc $B$ production vertex
measurement, based on $K^0_S$ vertex and $e^+e^-$ interaction point
constraint~\cite{Aubert:2005dy,Aubert:2007me}. These measurements
highly constrained the presence of non Standard Model processes with
large flavor violation.  I have been a collaborator of $BaBar$
experiment, running at the PEP-II $e^+e^-$ collider in the SLAC
laboratory (Stanford, California), from 2002 to
2009~\cite{seminario_ucsd,seminario_princeton,seminario_roma1_1,seminario_roma1_2}.

\vskip 0.5 truecm

From 2005 to 2006 operations manager and performance studies for the
muon and neutral hadron identification system of BaBar based on RPC
technology (IFR). Innovative algorithms for reconstructing and
identifying $K^0_L$ particles through the combined use of
electro-magnetic calorimeter (EMC) and IFR. Demonstration of the
aging effects of fluoridric acid (HF) production in the gas mixture
and possible mitigations of it~\cite{Band:2008zzb}. This detector has
shown a stable high efficiency through the entire data-taking period,
also with the change of the operating mode from {\it streamer} to {\it
  avalanche}~\cite{BandProc,Anulli:2005wi}. Upgrade of the barrel section of 
the IFR with Limited Streamer Tubes (LST).

\vskip 0.5 truecm

In 2005 and 2006 responsible for the $B$ flavour tagging efficiencies
and vertexing resolution, with the measurements used by all the BaBar
collaboration.  In 2005, tracking studies for the recovery of the loss
of efficiency in $K^0_S$ reconstruction after the shutdown, allowing a
complete recovery at the start of the new run period of BaBar.

\vskip 0.5 truecm

In 2007 I joined CMS collaboration and was involved on several
experimental activity regarding CMS commissioning: from 2007 to 2010
the high voltage system for the avalanche photo-diodes and development
of data quality monitoring of the electro-magnetic calorimeter
(ECAL)~\cite{ecal_craft}. On the software side, developer of the
online and offline ECAL Data Quality Monitoring (DQM) system, which is
still in use~\cite{proc_siena}.

From 2011 to 2014 responsible of the monitoring of
$PbWO_4$ crystals through a laser system, that is used to correct the
continuous transparency variations due to radiation during the LHC
fills.

\vskip 0.5 truecm

With the very early LHC data (36 pb$^{-1}$ at $\sqrt{s}$=7 TeV),
principal author of the first measurement of the cross section for $W$
and $Z$ production in association to up 4
jets~\cite{cms_vecbos_paper,proc_ifae,conf_ifae09,conf_lishep}.  With
the same data, measurement of the $W$ charge asymmetry, useful as
constraints for the PDFs~\cite{Chatrchyan:2011jz}.  Contribution to
the first measurement of inclusive $W$ and $Z$ production with the
early data, with the electron
channels~\cite{inclusiveV_2010,Khachatryan:2010xn,conf_plhc,proc_plhc}.

\vskip 0.5 truecm 

From 2008 developer of electron reconstruction and identification with
the CMS early data. From 2013 to 2014 coordinator of the electrons
and photons reconstruction sub-group of the EGamma physics
group~\cite{elepaper}.

\vskip 0.5 truecm

Since 2007 involved in the direct search of the Higgs boson of the
Standard Model
(SM)~\cite{cms_hww_paper2010,Chatrchyan:2012ty,Chatrchyan:2012tx,conf_hh}.
From 2011 to 2012 in $CMS$ experiment analysis and convenorship of the
physics group for the search of the Higgs boson with $W$ pairs, that
gave the first indication of the 125 GeV particle, and finally
contributed to its discovery~\cite{higgsdiscovery}. Presented
personally at the ICHEP 2012 conference in
Melbourne~\cite{conf_ichep12,proc_ichep12}. Proposed an analysis
method to allow the mass measurement in the presence of two neutrinos
up to 3\% precision~\cite{Chatrchyan:2013iaa}.
Measurements of the SM $pp\to W^+W^-$
production cross section at $\sqrt{s}$=7 and 8 TeV.

From 2012 to 2014 one of the principal authors of the Higgs search
with ZZ$\to4\ell$ final state (on electrons optimization for the
search and then on the properties measurement: mass and
spin-CP~\cite{conf_ichep14}). Coordinator of the analysis group and
the editor of the paper~\cite{h4l_legacy} on Run I data from 2013 to
2014~\cite{h4l_jcp,h4l_differential}. Developer of an analysis technique
for unfolding effective Higgs couplings from data through an 8D fit,
in collaboration with theorists~\cite{h4l_8d}.

\vskip 0.5 truecm

From 2014 to 2017 convener of the CMS ECAL Detector Performace Group
(DPG) as a group convener, responsible for the optimizations of its
running conditions, energy reconstruction algorithms,
calibrations~\cite{photonpaper}. In this period I covered the
transition from Run I to Run II, when LHC changed the bunch spacing
from 50 to 25 ns.  In 2014, main developer of the of the completely
new local energy reconstruction algorithm for
ECAL~\cite{conf_ieee14}. This allowed ECAL to maintain the optimal
energy resolution in conditions of high pileup.  The algorithm is
still in use and performing well up to the unprecedented peak pileup
of 60 events per bunch crossing~\cite{conf_hc16}. It is the standard 
algoithm foreseen for the HL LHC phase in the ECAL barrel.

\vskip 0.5 truecm

From 2015 to 2017 participated to the search of dark matter with MET
plus jets in the final state with 2016 CMS data (``monojet'' and
invisible Higgs production), also with the supervision of a master
degree student (Rome ``La Sapienza'')~\cite{monojet}.

\vskip 0.5 truecm

From 2017, author of the preparatory measurements needed for the $W$
mass measurement ($W$ helicity, rapidity, charge asymmetry, $Z$ and
$W$ $p_T$ measurements) with leptonic decays. Focus on both the
experimental aspects (optimization of lepton scale up to $10^{-4}$
precision), theoretical aspects (PDFs, QCD and EWK uncertainties) and
measurement aspects (global fitting procedure).

\vskip 0.5 truecm

From 2016 I am also involved in the development of a gas TPC prototype
for WIMPs detection using a combined electronics and optical readout
of GEM technology~\cite{pinci_jinst} (R\&D of the detectors and data analysis of the
test-beams performed in 2016 and 2017 at the BTF in Frascati).


\begin{center}
 { \bf Computing  Experience }
\end{center}

In-depth knowledge of C, C++, python, perl, shell scripting ({\it
  BASH}, {\it CSH}).


 
%%%%%%%%%%%%%%%%%%%%%%%%%%%
\newpage

\begin{center}
  {\bf Bibliometric parameters}
\end{center}

Parameters from http://inspirehep.net/ on 7 March 2018.

\vskip 0.4 truecm

\begin{tabular}{lcc}
Citation summary results &	Citeable papers	& Published only \\
Total number of papers analyzed  &	1,287 &	1,035 \\
Total number of citations  &	97,136 &	94,121 \\
Average citations per paper  &	75.5 &	90.9 \\
Breakdown of papers by citations  & & \\ 
Renowned papers (500+) &	16 &	15 \\
Famous papers (250-499) &	36 &	36 \\
Very well-known papers (100-249) &	183 &	181 \\
Well-known papers (50-99) &	270 &	262 \\
Known papers (10-49) &	485 &	420 \\
Less known papers (1-9) &	228 &	113 \\
Unknown papers (0) &	69 &	8 \\
hHEP index &	143 &	142 \\
\end{tabular}

\vskip 1.0 truecm

\begin{center}
  {\bf Selected publications}
\end{center}

{\it In the following a selection of publications in refereed
  international journals. To each of these publications I gave a
  relevant contribution.}

\vskip 1.0 truecm

\begin{thebibliography}{99999}
  \bibitem[*]{cmsphyspapers}
        {\bf CMS physics papers}
        \\    

\bibitem[pub38]{monojet} 
  A.~M.~Sirunyan {\it et al.} [CMS Collaboration],
  ``Search for dark matter produced with an energetic jet or a hadronically decaying W or Z boson at $ \sqrt{s}=13 $ TeV,''
  JHEP {\bf 1707}, 014 (2017)

\bibitem[pub37]{photonpaper} 
  V.~Khachatryan {\it et al.} [CMS Collaboration],
  ``Performance of Photon Reconstruction and Identification with the CMS Detector in Proton-Proton Collisions at sqrt(s) = 8 TeV,''
  JINST {\bf 10}, no. 08, P08010 (2015)

\bibitem[pub36]{elepaper} 
  V.~Khachatryan {\it et al.} [CMS Collaboration],
  ``Performance of Electron Reconstruction and Selection with the CMS Detector in Proton-Proton Collisions at √s = 8  TeV,''
  JINST {\bf 10}, no. 06, P06005 (2015)

\bibitem[pub35]{h4l_jcp} 
  V.~Khachatryan {\it et al.} [CMS Collaboration],
  ``Constraints on the spin-parity and anomalous HVV couplings of the Higgs boson in proton collisions at 7 and 8 TeV,''
  Phys.\ Rev.\ D {\bf 92}, no. 1, 012004 (2015)

\bibitem[pub34]{h4l_legacy} 
  S.~Chatrchyan {\it et al.} [CMS Collaboration],
  ``Measurement of the properties of a Higgs boson in the four-lepton final state,''
  Phys.\ Rev.\ D {\bf 89}, no. 9, 092007 (2014)

\bibitem[33]{h4l_differential} 
  V.~Khachatryan {\it et al.} [CMS Collaboration],
  ``Measurement of differential and integrated fiducial cross sections for Higgs boson production in the four-lepton decay channel in pp collisions at $ \sqrt{s}=7 $ and 8 TeV,''
  JHEP {\bf 1604}, 005 (2016)

\bibitem[pub32]{Chatrchyan:2013iaa} 
  S.~Chatrchyan {\it et al.} [CMS Collaboration],
  ``Measurement of Higgs boson production and properties in the WW decay channel with leptonic final states,''
  JHEP {\bf 1401}, 096 (2014)

\bibitem[pub31]{higgsprop}
  S.~Chatrchyan {\it et al.}  [CMS Collaboration],
  ``Study of the Mass and Spin-Parity of the Higgs Boson Candidate via Its Decays to Z Boson Pairs'',
  Phys.\ Rev.\ Lett.\  {\bf 110}, 081803 (2013)

\bibitem[pub30]{higgsdiscovery} 
  S.~Chatrchyan {\it et al.}  [CMS Collaboration],
  ``Observation of a new boson at a mass of 125 GeV with the CMS experiment at the LHC,''
  Phys.\ Lett.\ B {\bf 716}, 30 (2012)

\bibitem[pub29]{Chatrchyan:2012ty} 
  S.~Chatrchyan {\it et al.}  [CMS Collaboration],
  ``Search for the standard model Higgs boson decaying to a W pair in the fully leptonic final state in pp collisions at sqrt(s) = 7 TeV,''
  Phys.\ Lett.\ B {\bf 710}, 91 (2012)

\bibitem[pub28]{Chatrchyan:2012tx} 
  S.~Chatrchyan {\it et al.}  [CMS Collaboration],
  ``Combined results of searches for the standard model Higgs boson in pp collisions at sqrt(s) = 7 TeV,''
  Phys.\ Lett.\ B {\bf 710}, 26 (2012)

\bibitem[pub27]{Chatrchyan:2012dg} 
  S.~Chatrchyan {\it et al.}  [CMS Collaboration],
  ``Search for the standard model Higgs boson in the decay channel H to ZZ to 4 leptons in pp collisions at sqrt(s) = 7 TeV,''
  Phys.\ Rev.\ Lett.\  {\bf 108}, 111804 (2012)

\bibitem[pub26]{Chatrchyan:2011jz} 
  S.~Chatrchyan {\it et al.} [CMS Collaboration],
  ``Measurement of the lepton charge asymmetry in inclusive $W$ production in pp collisions at $\sqrt{s} = 7$ TeV,''
  JHEP {\bf 1104}, 050 (2011)

\bibitem[pub25]{cms_hww_paper2010}
  CMS~Collaboration,
  ``Measurement of W+W− production and search for the Higgs boson in pp collisions at sqrt(s) = 7 TeV''
  Physics Letters B {\bf 699}, 25 (2010)

\bibitem[pub24]{Chatrchyan:2011ek} 
  S.~Chatrchyan {\it et al.}  [CMS Collaboration],
  ``Inclusive search for squarks and gluinos in $pp$ collisions at $\sqrt{s}=7$ TeV,''
  Phys.\ Rev.\ D {\bf 85}, 012004 (2012)

\bibitem[pub23]{cms_vecbos_paper}
  CMS~Collaboration,
  ``Jet Production Rates in Association with W and Z Bosons in pp Collisions at sqrt(s) = 7 TeV''
  JHEP {\bf 1201}, 010 (2012)

\bibitem[pub22]{Khachatryan:2010xn} 
  V.~Khachatryan {\it et al.}  [CMS Collaboration],
  ``Measurements of Inclusive W and Z Cross Sections in pp Collisions at sqrt(s)=7 TeV,''
  JHEP {\bf 1101}, 080 (2011)

\bibitem[pub21]{inclusiveV_2010} 
  S.~Chatrchyan {\it et al.} [CMS Collaboration],
  ``Measurement of the Inclusive $W$ and $Z$ Production Cross Sections in $pp$ Collisions at $\sqrt{s}=7$ TeV,''
  JHEP {\bf 1110}, 132 (2011)


\vskip 1.0 truecm

\bibitem[*]{cmsdetpapers}
{\bf CMS detector papers}
\\

\bibitem[pub20]{cms_craft}
  CMS~Collaboration,
  ``Commissioning of the CMS experiment and the cosmic run at four tesla'',
  JINST {\bf 5} (2010) T03001

\bibitem[pub19]{ecal_craft}
 CMS~Collaboration
 ``Performance and operation of the CMS electromagnetic calorimeter''
 JINST {\bf 5} (2010) T03010

\bibitem[pub18]{time_craft}
 CMS~Collaboration
 `` Time reconstruction and performance of the CMS electromagnetic calorimeter''
 JINST {\bf 5} (2010) T03011

%\cite{:2008zzk}
\bibitem[pub17]{longo_cms}
  R.~Adolphi {\it et al.}  [CMS Collaboration],
  ``The CMS experiment at the CERN LHC,''
  JINST {\bf 3} (2008) S08004





\vskip 1.0 truecm

\bibitem[*]{babarphyspapers}
{\bf BaBar physics papers}

%\cite{Aubert:2008gy}
\bibitem[pub16]{Aubert:2008gy}
  B.~Aubert {\it et al.}  [BABAR Collaboration],
  ``Measurement of Time-Dependent CP Asymmetry in $B^0 \to K^0_{S} \pi^0
  \gamma$ Decays,''
  Phys.\ Rev.\  D {\bf 78} (2008) 071102
  %%CITATION = PHRVA,D78,071102;%%


%\cite{Aubert:2007sd}
\bibitem[pub15]{Aubert:2007sd}
  B.~Aubert {\it et al.}  [BABAR Collaboration],
  ``Measurements of CP-Violating Asymmetries in the Decay $B^0 \to K^+K^-K^0$''
  Phys.\ Rev.\ Lett.\  {\bf 99}, 161802 (2007)
%  [arXiv:0706.3885 [hep-ex]].
  %%CITATION = PRLTA,99,161802;%%

%\cite{Aubert:2007me}
\bibitem[pub14]{Aubert:2007me}
  B.~Aubert {\it et al.}  [BABAR Collaboration],
  ``Measurement of CP Asymmetries in $B^0 \to K^0_S K^0_S K^0_S$ Decays'',
  Phys.\ Rev.\  D {\bf 76}, 091101 (2007)
%  [arXiv:hep-ex/0702046].
  %%CITATION = PHRVA,D76,091101;%%

%\cite{Aubert:2007mj}
\bibitem[pub13]{Aubert:2007mj}
  B.~Aubert {\it et al.}  [BABAR Collaboration],
  ``Observation of CP violation in $B^0 \to K^{+} \pi^{-}$ and $B^0 \to \pi^{+} \pi^{-}$,''
  Phys.\ Rev.\ Lett.\  {\bf 99}, 021603 (2007)
%  [arXiv:hep-ex/0703016].
  %%CITATION = PRLTA,99,021603;%%

%\cite{Aubert:2007mgb}
\bibitem[pub12]{Aubert:2007mgb}
  B.~Aubert {\it et al.}  [BABAR Collaboration],
  ``Measurement of the CP-violating asymmetries in $B^0 \to K^0_{S} \pi^0$ and of the branching fraction of $B^0 \to K^0 \pi^0$,''
  Phys.\ Rev.\  D {\bf 77}, 012003 (2008)
  %%[arXiv:0707.2980 [hep-ex]].
  %%CITATION = PHRVA,D77,012003;%%

\bibitem[pub11]{Aubert:2007hm} 
  B.~Aubert et al. [BABAR Collaboration] 
  {\it ``Improved measurement of CP violation in neutral B decays to $c \bar{c} s$''},
  Phys.\ Rev.\ Lett.\  {\bf 99}, 171803 (2007)
 % Arxiv:hep-ex/0703021.

%\cite{Aubert:2006nn}
\bibitem[pub10]{Aubert:2006nn}
  B.~Aubert {\it et al.}  [BABAR Collaboration],
  ``Search for $B^+ \to \phi \pi^+$ and $B^0 \to \phi \pi^0$ decays,''
  Phys.\ Rev.\ D {\bf 74}, 011102 (2006)
%  [arXiv:hep-ex/0605037].
  %%CITATION = HEP-EX 0605037;%%

%\cite{Aubert:2006nu}
\bibitem[pub9]{Aubert:2006nu}
  B.~Aubert {\it et al.}  [BABAR Collaboration],
  ``Dalitz plot analysis of the decay $B^{\pm} \to K^{\pm} K^{\pm} K^{\mp}$,''
  Phys.\ Rev.\  D {\bf 74}, 032003 (2006)
%  [arXiv:hep-ex/0605003].
  %%CITATION = PHRVA,D74,032003;%%


%\cite{Aubert:2006wv}
\bibitem[pub8]{Aubert:2006wv}
  B.~Aubert {\it et al.}  [BABAR Collaboration],
  ``Observation of CP violation in $B^0 \to \eta^\prime K^0$ decays,''
  Phys.\ Rev.\ Lett.\  {\bf 98}, 031801 (2007)
  %%[arXiv:hep-ex/0609052].
  %%CITATION = PRLTA,98,031801;%%

%\cite{Aubert:2006zy}
\bibitem[pub7]{Aubert:2006zy}
  B.~Aubert {\it et al.}  [BABAR Collaboration],
  ``Search for the decay $B^0 \to K^0_{s} K^0_{s} K^0_{L}$,''
  Phys.\ Rev.\  D {\bf 74}, 032005 (2006)
  %%[arXiv:hep-ex/0606031].
  %%CITATION = PHRVA,D74,032005;%%

%\cite{Aubert:2006gm}
\bibitem[pub6]{Aubert:2006gm}
  B.~Aubert {\it et al.}  [BABAR Collaboration],
  ``Observation of $B^{+} \to \bar{K}^0 K^{+}$ and $B^0 \to K^0 \bar{K}^0$,''
  Phys.\ Rev.\ Lett.\  {\bf 97}, 171805 (2006)
  %%[arXiv:hep-ex/0608036].
  %%CITATION = PRLTA,97,171805;%%

%\cite{Aubert:2005ja}
\bibitem[pub5]{Aubert:2005ja}
  B.~Aubert {\it et al.}  [BABAR Collaboration],
   ``Measurement of CP asymmetries in $B^0 \to \phi K^0$ and $B^0 \to K^+ K^-K^0_S$  decays,''
  Phys.\ Rev.\ D {\bf 71}, 091102 (2005)
%  [arXiv:hep-ex/0502019].
  %%CITATION = HEP-EX 0502019;%%

%\cite{Aubert:2005dy}
\bibitem[pub4]{Aubert:2005dy}
  B.~Aubert {\it et al.}  [BABAR Collaboration],
  ``Branching fraction and CP asymmetries of $B^0 \to K^0_S K^0_S K^0_S$'', 
  Phys.\ Rev.\ Lett.\  {\bf 95}, 011801 (2005)
%  [arXiv:hep-ex/0502013].
  %%CITATION = HEP-EX 0502013;%%


\vskip 1.0 truecm

\bibitem[*]{babardetpapers}
{\bf BaBar detector papers}

%\cite{Anulli:2005wi}
\bibitem[pub3]{Anulli:2005wi}
  F.~Anulli {\it et al.},
  ``Performance of 2nd generation BaBar resistive plate chambers,''
  Nucl.\ Instrum.\ Meth.\ A {\bf 552}, 276 (2005)

%\cite{Band:2008zzb}
\bibitem[pub2]{Band:2008zzb}
  H.~R.~Band {\it et al.},
  ``Study Of HF Production In BaBar Resistive Plate Chambers'',
  Nucl.\ Instrum.\ Meth.\  A {\bf 594}, 33 (2008).

\bibitem[pub1]{BandProc}
  H.~R.~Band {\it et al.},
  ``Performance and Aging Studies of BaBar Resistive Plate Chambers''
  Nucl.\ Physics B {\bf 158}, 139-142 (2006).

\vskip 1.0 truecm

\bibitem[*]{otherpapers}
{\bf Other papers}

\bibitem[opub2]{pinci_jinst}
  C.~Antochi~Vasile {\it et al.},
  ``Combined readout of a triple-GEM detector''
  Submitted to JINST (2018)

\bibitem[opub1]{h4l_8d}
  Y.~Chen, E.~Di Marco, J.~Lykken, M.~Spiropulu, R.~Vega-Morales and S.~Xie,
  ``8D likelihood effective Higgs couplings extraction framework in $h \to 4\ell$,''
  JHEP {\bf 1501}, 125 (2015)



\vskip 1.0 truecm

  
  \begin{center}
  \bibitem{crscelti}
    {\bf Other publications and selected conference reports }
    \\
  \end{center}
  \bibitem[*]{cmsproc}
{\bf CMS proceedings and relevant public notes}
\\

\bibitem[p9]{proc_ichep12} 
  E.~Di Marco [CMS Collaboration], 
  ``Search for SM Higgs decaying to WW $\to\ell\nu\ell\nu$ and $\ell\nu$qq at CMS'', 
  PoS ICHEP {\bf 2012}, 076 (2013)

\bibitem[p8]{proc_ifae2012}
  E.~Di Marco [CMS Collaboration],
  ``Searches for the standard model Higgs boson at CMS'',  
  Il Nuovo Cimento {\bf 36}, 1 (2013)

\bibitem[p7]{Chatrchyan:2013oev} 
  S.~Chatrchyan {\it et al.}  [ CMS Collaboration],
  ``Measurement of W+W- and ZZ production cross sections in pp collisions at sqrt(s)=8 TeV,''
  CERN-PH-EP-2012-376, arXiv:1301.4698 [hep-ex].
  %%CITATION = ARXIV:1301.4698;%%
%\cite{:2013qh}

\bibitem[p6]{proc_plhc} 
  E.~Di Marco [CMS Collaboration],
  ``Observation of W and Z boson candidates with the CMS experiment'',
  DESY-PROC-2010-01.
  %%CITATION = DESY-PROC-2010-01;%%
%\cite{Chatrchyan:2011dx}

\bibitem[p5]{proc_ifae} E.~Di Marco ``Measurement of W and Z
  production in association with jets with CMS detector'' Il Nuovo
  Cimento {\bf 32}, 3 (2009)

\bibitem[p4]{proc_siena}
  E.~Di Marco,
  ``The CMS ECAL data quality monitoring and first results with cosmics data'',
  Nucl.\ Physics B {\bf 197}, 267-270 (2009)



\bibitem[*]{babarproc}
{\bf BaBar proceedings and relevant public notes}

%\cite{Bona:2007qt}
\bibitem[p3]{Bona:2007qt}
  M.~Bona {\it et al.},
  ``SuperB: A High-Luminosity Asymmetric e+ e- Super Flavor Factory. Conceptual Design Report,''
  SLAC-R {\bf 856}, INFN-AE {\bf 07-02}, LAL {\bf 07-15}
  %%CITATION = ARXIV:0709.0451;%%

%\cite{DiMarco:2006wg}
\bibitem[p2]{DiMarco:2006wg}
  E.~Di Marco  [BABAR Collaboration],
  ``Measurement of direct CP asymmetries in charmless hadronic B decays,''
  Moscow 2006, ICHEP 843-850 

%\cite{DiMarco:2007af}
\bibitem[p1]{DiMarco:2007af}
  E.~Di Marco,
  ``Measurement of angle $\beta$ with time-dependent CP asymmetry in $B^0 \to K^+ K^-K^0$ decays,''
  proceeding for an invited talk at 4th International Workshop on the 
  CKM Unitarity Triangle (CKM 2006), Nagoya, Japan, 12-16 Dec 2006.







  \newpage

  \begin{center}
  \bibitem{conf}
    {\bf International Conferences talks}
    \\
  \end{center}
  \bibitem[ci12]{conf_hc16}
  {\bf ``Role of the CMS electromagnetic calorimeter in the measurement of the Higgs boson properties''}
  talk at conference ``Higgs Couplings 2016'', SLAC - Stanford (USA), November 2016.

\bibitem[ci11]{conf_ichep14}
  {\bf ``Studies of the Higgs boson spin and parity using the gamma gamma, ZZ, and WW decay channels with the CMS detector}
  talk at conference ``ICHEP 2014'', Valencia (Spain), July 2014.

\bibitem[ci10]{conf_ieee14}
  {\bf ``CMS electromagnetic calorimeter calibration and timing         
  performance during LHC Run I and future prospects'}
  talk at conference ``IEEE 2014'', Seattle (USA), November 2014.

\bibitem[ci9]{conf_ichep12}
  {\bf ``Search for Higgs in WW decays at CMS''}
  talk at conference ``ICHEP 2012'', Melbourne (Australia), July 2012.

\bibitem[ci8]{conf_hh}
  {\bf ``Higgs into WW and ZZ at CMS''}
  talk at conference ``Higgs Hunting 2011'', Paris (Fance), July 2011.

\bibitem[ci7]{conf_lishep}
  {\bf ``V+ Jets and V+gamma at the LHC''}
  talk at conference ``Lishep 2011'', Rio de Janeiro (Brasil), July 2011.

\bibitem[ci6]{conf_plhc}
  {\bf ``Observation of W and Z production with CMS experiment'' }
  talk at conference ``Physics at LHC 2010'' Hamburg (Germany), June 2010.

\bibitem[ci5]{conf_siena}
  {\bf ``The CMS ECAL data quality monitoring and first results with cosmics data''}
  talk at conference ``$11^{th}$ Topical Seminar on Innovative Particle and Radiation Detectors'',
  Siena (Italy), October 2008.

\bibitem[ci4]{talk_ichep}
  {\bf ``Direct CP Asymmetries in Charmless B Decays with the BaBar experiment''}
  International Conference on High Energy Physics (ICHEP2006), Moscow (Russia), July 2006
  \vspace{3mm}

\bibitem[ci3]{talk_nagoya}
  {\bf ``Measurement of CKM angle $\beta$ with time dependent Dalitz plot analysis of 
    $B^0\rightarrow K^+K^-K^0$ decays''}
  IV CKM Workshop, Nagoya (Giappone), December 2006

\bibitem[ci2]{conf_win05}
  {\bf ``Measurement of $\sin 2\beta$ at B-factories''}
  Worksop Weak Interactions and Neutrinos (WIN 2005), Delphi (Grecia), June 2005
  \vspace{3mm}

\bibitem[ci1]{conf_aps}
  {\bf ``Measurement of CP asymmetries in $b \rightarrow s$ decays at $BaBar$''}
  Annual Meeting of the American Physical Society, Tampa (FL), April 2005
  \vspace{3mm}




  \begin{center}
  \bibitem{confita}
    {\bf Italian Conferences talks}
    \\
  \end{center}
  \bibitem[cn3]{conf_ifae12}
  {\bf ``Ricerca di Higgs a CMS''}
  plenary talk at conference ``Incontri di Fisica delle Alte Energie - XI edizione'',
  Ferrara (Italy), April 2012.

\bibitem[cn2]{conf_ifae09}
  {\bf ``Misura di W e Z con produzione associata di jet a CMS''}
  talk at conference ``Incontri di Fisica delle Alte Energie - VIII edizione'',
  Bari (Italy), April 2009.

\bibitem[cn1]{talk_lnf}
  {\bf ``Measurement of CP asymmetry in $b \rightarrow s$ decays''},
  plenary talk the the ``LNF spring school'', Roma (Italy), April 2004
  \vspace{3mm}

  
  \begin{center}
  \bibitem{seminari}
    {\bf Seminars}
    \\
  \end{center}
    % seminari
\bibitem[s5]{seminario_caltech}
  {\bf ``Higgs searches with diboson decays with CMS experiment''}
  High Energy Phyisics Seminar given at California Institute of Technology, Pasadena (USA), November 2011

\bibitem[s4]{seminario_ucsd}
  {\bf ``Search for new physics with time-dependent CP asymmetries in $b\to s$ transitions''}
  High Energy Phyisics Seminar given at University of California, San Diego (USA), December 2006

\bibitem[s3]{seminario_princeton}
  {\bf ``Search for new physics with time-dependent CP asymmetries in $b\to s$ transitions''}
  High Energy Phyisics Seminar given at Princeton University (USA), January 2007

\bibitem[s2]{seminario_roma1_2}
  {\bf ``Search for physics beyond the Standard Model at B-factories''}
  Seminar of Particles and Fields,  Universit\`a di Roma ``La Sapienza'', Roma (Italy), May 2006
  \vspace{3mm}

\bibitem[s1]{seminario_roma1_1}
  {\bf ``Flavour Changing Neutral Current processes in $B$ decays''}
  Seminar of Particles and Fields, University of Rome ``La Sapienza'', Roma (Italy), November 2004
  \vspace{3mm}


\end{thebibliography}



\end{document}


