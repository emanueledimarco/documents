\documentclass[11pt,twoside,a4paper]{article}
%\usepackage[english,italian]{babel}
\usepackage{latexsym}
\usepackage{times}
\newcommand{\HRule}{\rule{\linewidth}{0.2mm}}
\usepackage{fancyheadings}
\thispagestyle{empty}
\setlength{\parindent}{0mm}
\setlength{\parskip}{0mm}
\usepackage[dvips]{graphicx}
\usepackage{fontenc}
\usepackage{eucal}
\usepackage{amsfonts}
\usepackage{epic,rotating,epsfig}
\usepackage{verbatim}
\usepackage{syntonly}
\usepackage{amssymb}
\usepackage{amsmath}
\usepackage{cite}
\pagestyle{empty}
%%%%%%%%%%%%%%%%%%%%%%%%%%%
\hoffset = -2.0 cm
\voffset = -3.0 cm
\textheight = 25.0 cm
\marginparwidth = 1.0 cm
\evensidemargin = 1.0 cm
\textwidth =16.1 cm
%%%%%%%%%%%%%%%%%%%%%%%%%%%


\def\bibname{ } 
\def\nextref{\global\advance\refno by 1 \number\refno\relax}
%
\begin{document}


\large
\centerline{\bf Curriculum Vit\ae{} of Emanuele Di Marco}
\vskip 0.5 truecm


\begin{itemize} 
\item  Name:  Di Marco
\item  First Name: Emanuele
\item  Born: February, $7^{th}$, 1979, Roma (Italy)
\item  Citizenship: Italian
\item  Marital status: single
\item  Professional address: CERN, Gen\'eve CH-23, Switzerland
\item  Private address: via Nostra Signora di Lourdes, 126, 00167 Roma (Italy)
\item  Phone: +41 76 226 1272
\item  e-mail: dimarcoe@caltech.edu
\item  Languages: Mother tongue Italian. Speaks, reads, and writes fluently English. 
\end{itemize}

\begin{center}
{\bf{Education}}
\end{center}
\begin{itemize}

\item May 1, 2014 - April 30, 2017: {\it ``Marie Curie''} Fellow
  (COFUND) for {\bf CERN, European Organization for Nuclear Research},
  Geneva, Switzerland, and member of $CMS$ Collaboration.

\item August 1, 2011 - April 30, 2014: {\it ``Tolman Fellow''} for {\bf
  California Institute of Technology}, Pasadena, USA, and
  member of $CMS$ Collaboration.

\item September 1, 2009 - July 31, 2011: postdoctoral research
  associate for the {\bf ``La Sapienza Universit\`a di Roma''} as a member
  of the $CMS$ Collaboration with research title: {\it ``Search for
    di-electron resonances not foreseen by the Standard Model with the
    first LHC data''}.

\item May 1, 2008 - April 30, 2009: {\bf INFN fellowship at CERN} in
  the framework of LHC experiments with the research theme: {\it
    ``Commissioning of data quality monitoring and low-level
    calibrations of CMS electro-magnetic calorimeter with cosmic data
    and first collision data''}

\item June 1, 2007 - May 31, 2009: postdoctoral research associate for
  the {\bf Universit\`a ``La Sapienza'' in Rome} as a member of the
  $CMS$ Collaboration with research title: {\it ``Di-electron
    resonances at LHC: data interpretation in theoretical frameworks
    beyond the Standard Model''}.

\item 2003-2006 Ph.D. in Physics at the Universit\`a ``La Sapienza''
  in Rome.\\ Thesis in Particle Physics defended on $23^{th}$ January
  2007, with title: {\it Measurements of $CP$ violating asymmetries in
    charmless three-body $B$ decays with the $BaBar$ experiment},
  supervisor Prof. F.~Ferroni.

\item $17^{th}$ July 2003 Graduation in Particle Physics at the
  Universit\`a ``La Sapienza'' in Rome with full marks (110/110), with
  the thesis in experimental Particle Physics: {\it Study of the
    decays $B^0 \rightarrow \phi K^0$ with the $BaBar$ experiment},
  supervisor Prof. F. Ferroni.

\item 1998-2002  Physics student at  Universit\`a  "La Sapienza" in Rome.

\item 1998    Diploma of scientific secondary studies with full marks (60/60),
  at Liceo Scientifico Statale Talete in Rome.

\end{itemize}

\begin{center}
  {\bf{Grants}}
\end{center}

\begin{itemize}
\item 2009: ``CMS Achievement Award''. Award for outstanding
  contribution in the $CMS$ $ECAL$ commissioning through the development
  of the $ECAL$ high voltage system (hardware) and data quality
  monitoring (software).
\end{itemize}

\clearpage

\begin{center}
{\bf{Research activity}}
\end{center}
My research activity focused on the experimental high energy physics.
It started with the study of CP violation in the $B$ meson weak decays
and the search of indirect signs of physics beyond the Standard Model
in the flavor sector through charmless $b \to s$ decays. I measured
the time-dependent CP violation of the $B$ meson in three kaon decays,
including interference effects through the Dalitz plot. These
measurements highly constrained the presence of non Standard Model
processes with large flavor violation.  I have been a collaborator of
$BaBar$ experiment, running at the PEP-II $e^+e^-$ collider in the
SLAC laboratory (Stanford, California), from 2002 to 2009.

\vskip 0.5 truecm

When I was a $BaBar$ collaborator, together with the activity in the
data analysis, I have been involved in the commissioning and
performance studies for the Resistive Plate Chambers (RPC) in the
forward endcap of the $BaBar$ detector. I have demonstrated the aging
effects of fluoridric acid production in the gas mixture and possible
mitigations of it.  This detector has shown a stable high efficiency
through the entire data-taking period.

\vskip 0.5 truecm

Since 2007 my interest moved to experiments involved in the
understanding of the electroweak symmetry breaking mechanism through
the direct search of the Higgs boson of the Standard Model.  In $CMS$
experiment I had a leading role, through the analysis and convenorship
of the physics group for the search of the Higgs boson decaying into
$W$ pairs, during the first two years of data-taking of the pp Large
Hadron Collider (LHC) at European Laboratory for Particle Physics
(CERN) in Geneva.  This challenging channel contributed to the
discovery of a boson with mass around 125 $GeV$ in July 2012, and
currently provides the most precise measurement of the signal
strength.  On the same line of research, since 2012 I am one of the
principal authors of the Higgs search through decays into a pair of
$Z$ bosons, with $4\ell$ final state. My personal contributions have
been especially on the electrons optimization and on the mass
measurement: this led to the most precise measurement of the mass of
the newly discovered particle. The measurements of these two decay
modes measurements have been done with 24.1 fb$^{-1}$ of data
collected during Run I.

\vskip 0.5 truecm

With the very early LHC data (36 pb$^{-1}$ at $\sqrt{s}$=7 TeV) I have
been the principal author of the analysis for the direct
detection of non Standard Model physics signals through the production
of electroweak bosons $W$ and $Z$ in association with jets at the
LHC. Having observed the absence of non Standard Model contribution, I
have performed the first measurement of the cross section for $W$ and
$Z$ production in association to up 4 jets in $pp$ collisions at
$\sqrt{s}$= 7 TeV.

\vskip 0.5 truecm

In the period of INFN fellowship (CERN associate) I have been focused
on experimental activity regarding CMS commissioning: high voltage
system for the avalanche photo-diodes and development of data quality
monitoring of the electro-magnetic calorimeter (ECAL). I am currently
involved in the ECAL performance through the operation of the
monitoring of $PbWO_4$ crystals through a laser system, that is used
to correct the continuous transparency variations due to radiation
during the LHC fills.

\vskip 0.5 truecm

I am one of the main developers of electron reconstruction and
identification in CMS since the early data, and currently I am
the coordinator of the electrons and photons reconstruction group.

\vskip 0.5 truecm

Starting from January 2014 I will lead the CMS ECAL Detector
Performace Group (DPG) as a group convener.

\vskip 1.5 truecm



\newpage

\begin{center}
  {\bf CMS experiment at the LHC collider at CERN}
\end{center}

\begin{enumerate}
\item {\bf Research responsibilities and contributions to detector}
  
  \begin{itemize}
    \item January 2014 - December 2016: {\bf convener of the CMS ECAL
      Detector Performance Group}.

  \item March 2013 - now: coordinator of the $H \to ZZ\to4\ell$
    analysis and {\bf editor} of the CMS legacy paper on LHC Run1 data

  \item January 2013 - now: member of ECAL institution board
    
  \item January 2013 - December 2014: {\bf group convener} of the
    electron and photon reconstruction group
    
  \item 2011 - 2012: {\bf group convener} for the Higgs boson search
    in W pair decays, one of the key decay modes for the Higgs search
    in a very wide range of masses from 120 to 600 GeV/c$^2$. 

  \item I am the {\bf co-editor} of the review book {\it The search
    for the Standard Model Higgs Boson}, documenting the studies done
    at high energy colliders (LEP, Tevatron and LHC) for the search of
    the Higgs boson. It will be published by World Scientific in
    Summer 2013.
    
  \item August 2011-now: responsible for the {\bf laser monitoring
    system of the electro-magnetic calorimeter (ECAL)}. The 76,000
    lead tungstate ($PbWO_4$) crystals in the ECAL suffer from dose
    rate dependent radiation damage. The ability to calibrate the
    crystal calorimeter precisely and continuously during the
    data-taking is a major factor in determining its ultimate
    resolution, and regulates the Higgs sensitivity in the diphoton
    channel.

  \item 2009-2011: {\bf responsible for the ECAL High Voltage
    system}. I have done the installation, calibration and noise
    screening in situ of the high voltage system of the
    electro-magnetic calorimeter.  This system gives the bias to the
    photo-detectors ({\it Avalanche Photo Diodes}) whose gain depends
    highly on the bias voltage, so an excellent stability is required
    in order not to affect the energy resolution. The system worked as
    foreseen in the first long data-taking period~\cite{ecal_craft}.

  \item 2009-2011: I am the {\bf developer of online and offline ECAL
    Data Quality Monitoring (DQM)}. It has been used during the
    detector commissioning and now it is an integrating part of the
    online data taking and offline data quality assurance.  The DQM
    framework runs on the High Level trigger farm as well as on local
    data acquisition (DAQ) systems, making the results of pre-defined
    data analyses immediately available, and stores its output into
    the database~\cite{cms_craft}. Invited talk at IPRD
    2008~\cite{proc_siena}.
  \end{itemize}

\item {\bf Data analysis}
  \begin {itemize}
    
  \item {\bf Main author} of analysis, on Monte Carlo simulations and
    on the LHC Run1 data (4.6 fb$^{-1}$ of p-p collisions collected in
    2011 at $\sqrt{s}$=7 TeV and 19.7 fb$^{-1}$ at $\sqrt{s}$=8 TeV),
    for the search of the {\bf Higgs boson in the mass range 120--600
      GeV/c$^2$ through decays in two $W$s in fully leptonic final
      states}.  This is one of the main channels contributing to the
    discovery of the boson with mass around 125 $GeV$ in July
    2012. The observed excess of events above background is consistent
    with the expectations from a SM Higgs boson of mass 125 GeV and
    has a statistical significance of 3.1 standard deviations for this
    mass. This provides, to date, the most precise measurement of the
    signal strength. \cite{pas_HWW5,AN_2013_378}.  I have presented
    the result at ICHEP 2012, Melbourne,
    Australia~\cite{conf_ichep12,proc_ichep12}. It has been published in
    ~\cite{higgsdiscovery}.  It provided exclusion for fourth
    generation of quarks since the first result on 36 pb$^{-1}$ of
    2010 data. The result was published in~\cite{cms_hww_paper2010},
    and has been presented by me at~\cite{conf_hh}.  The intermediate
    result on 4.6 fb fb$^{-1}$ of 2011 data at 7 TeV has been
    published in~\cite{Chatrchyan:2012ty,Chatrchyan:2012tx}.  I also
    applied to the analysis the use of innovative kinematic variables
    ({\it razor}) to improve the mass sensitivity of this channel
    where the presence of two neutrinos does not allow to close the
    kinematics.  This allowed the measurement of the mass of the Higgs
    in the fully leptonic final state with 5 GeV precision, that
    becomes 4 GeV with the constraint of the SM cross section
    \cite{AN_2013_002,AN_2012_071}. The analysis on the full LHC Run1
    dataset has been submitted for
    publication~\cite{Chatrchyan:2013iaa}.
    
  \item {\bf Main author} of the measurement of $pp\to W^+W^-$
    production cross section at $\sqrt{s}$=7 and 8 TeV, as a side
    measurement of the Higgs analysis. The extra contributions, with
    respect to the study for the Higgs search, have been the
    minimization of the systematic uncertainties limiting the
    measurement up to now (jet counting efficiency and lepton-fakes
    contribution)~\cite{pas_WW1,pas_WW2}. Member of the Ph.D. thesis
    committee of a student of ``IFCA, Instituto de F\'isica de
    Cantabria'', Santander, Spain (C. Jord\'a Lope).
    
  \item {\bf Main author} of the analysis, with 2011 and 2012 data,
    since January 2012, of the analysis for the search of the {\bf
      Higgs boson in the mass range 110--1000 GeV/c$^2$ through decays
      in two $Z$s in $4\ell$ final state}. I am the editor of the
    paper with full LHC Run1 dataset.  My personal contributions in
    this analysis have been:
    \begin{enumerate}
    \item measurement of the Higgs mass, and direct constraint on the
      width, by developing the electron corrections and performing the
      analysis with the event-by-event mass uncertainties;
    \item re-definition of the electron identification in the low
      $p_T$ range (7$<p_T<$20 GeV), that strongly enhanced the
      sensitivity for a Higgs search in the range $m_H<$130 GeV (the
      only allowed region after the search with 2011 data);
    \item a new method to correct the energy measurement of the
      electrons through a multivariate energy regression, that
      improved the resolution of the $4e$ and $2e2\mu$ channels of
      about 15\% (limiting the Higgs mass systematics to only 0.3\%
      and 0.1\%, respectively). Results are being published
      \cite{pas_HZZProp}.
    \item signal lineshape modeling that allowed a uniform approach in
      the wide range of the search;
    \item measurement of the electron momentum scale linearity,
      through di-electron resonances ($J/\psi$, $\Upsilon$s, Z) to
      assign correct systematics for the Higgs mass measurement. This
      contributed to the best mass measurement of this particle, to
      date.
    \end{enumerate}
    The search is published in~\cite{higgsdiscovery}, and the updated
    measurement, with 17 fb$^{-1}$ provides an observation of a
    resonance with this channel alone, at the level of 5 standard
    deviations~\cite{pas_HZZ2}.  I presented the status of all the
    Higgs searches in an invited plenary talk at IFAE 2012 (Ferrara,
    Italy)~\cite{conf_ifae12,proc_ifae2012}.

  \item {\bf Author} of the inclusive search for squarks and gluinos
    through razor variables. Contributed to the search with the
    definition of the selection and calibration of the main physics
    objects considered: electrons, muons and jets. This analysis is
    one of the most sensitive ones in the search for SUSY and
    dark-matter. No signal has been detected in the LHC data analyzed
    so far. Results with 7 TeV data have been published
    in~\cite{Chatrchyan:2011ek}, and the update with full 8 TeV is
    foreseen for the 2013 Winter conferences.

  \item {\bf Main author} of the study of associated production of $W$
    and $Z$ bosons ($V$) with jets. The cross section as a function of
    the jet multiplicity, as well the cross section ratios $V$ + $n$
    jet over $V$+$(n+1)$ jet, for $n \ge 1$, is a test of QCD.  Since
    many processes beyond the Standard Model, as supersymmetry,
    foresee production of vector bosons with high jet multiplicity,
    this observable at high $n$ is a powerful handle for
    model-independent searches~\cite{proc_ifae,pas_Zjets,pas_Wjets}.
    Invited talk at IFAE 2009 (Bari, Italy) on the feasibility study
    with 100 pb$^{-1}$ at $\sqrt{s}=10$
    TeV~\cite{proc_ifae,conf_ifae09}.  The measurement has been
    performed on the first 36 pb$^{-1}$ of data, and published
    in~\cite{cms_vecbos_paper} and presented by me
    at~\cite{conf_lishep}. The charge asymmetry in $W^\pm$ + jets is
    sensitive to the parton density functions, especially at high
    $\eta$. 
    %% I have been {\bf Supervisor} of a master thesis on this
    %% subject with a student in ``La Sapienza Universit\`a di Roma''
    %% (F. Micheli).
    
  \item {\bf Main author of electron identification in CMS}, with a
    work on definition of identification through ECAL cluster shapes,
    tracker variables, and their combination, and its optimization on
    data control samples on 2011 and 2012 dataset. I have optimized
    and evaluated the performances on
    data~\cite{pas_idelectrons,AN_2009_178}, including the mitigation
    of pileup effects.

  \item {\bf Responsible of electron efficiencies} for the first
    measurement of inclusive W and Z production cross section in 7 TeV
    p-p collisions.  Invited talk at ``Physics at LHC
    2010''~\cite{conf_plhc,proc_plhc}. {\bf Advisor} of a master
    thesis on this subject with a student in ``Universit\`a degli
    Studi di Trieste'' (A. Schizzi).

  \item Definition the selection of a {\bf pure sample of $J/\psi\to
    e^+e^-$} in p-p collisions at 7 TeV, focusing on the
    reconstruction and identification of $p_T<$ 10 GeV/c electrons in
    $CMS$ and on the calibration of the absolute energy scale of
    $ECAL$. 
    %% {\bf Supervisor} of master thesis on this subject with a
    %% student in ``La Sapienza Universit\`a di Roma'' (S. Nourkhbash).

  \item {\bf Author of electron reconstruction in CMS}, with a work on
    the seeding in the pixel detector through association with ECAL
    clusters. I have also refined the pure-tracker seeding needed for
    low $p_T$ electrons, which play a role in the leptonic decay modes
    of Higgs boson to Z pair~\cite{pas_recoelectrons,AN_2009_164}.
    This has been validated on 7TeV data through $J/\psi$ di-electron
    resonance~\cite{AN_2010_443}. Since January 2013,
    coordinator of the group.
    
  \end{itemize}
\end{enumerate}



\begin{center}
  {\bf BaBar experiment at the PEP-II B-factory at SLAC}
\end{center}

\begin{enumerate}

\item {\bf Main author measurements}

\item As Ph.D. student, my interest has focused on the measurement of
  {\bf $CP$ violation in the weak decays of $B$ meson in charmless
    final states.} $b\to s$ elementary transitions, proceeding through
  loop amplitudes, have the maximal sensitivity to virtual New Physics
  particles affecting the measured $CP$ violation.  {\bf I am the main
    author} of the measurement of time-dependent $CP$ asymmetry
  simultaneous to the Dalitz plot analysis of the three body decay
  $B\to K^+ K^- K^0$, with $K^0=K^0_S, K^0_L$~\cite{Aubert:2007sd}.
  The results, compatible with the Standard Model, constrain the New
  Physics models with large flavor violation.  A new spin-zero
  resonance decaying in two charged kaons with mass around 1.55
  GeV/c$^2$, the $X_0(1550)$, have been found as contributing to the
  Dalitz plot.  Invited talk at CKM workshop 2006~\cite{talk_nagoya},
  and two experimental seminars, at University of California, San
  Diego~\cite{seminario_ucsd} and Princeton
  University~\cite{seminario_princeton}.

\item {\bf Reconstruction and identification of $K^0_L$} through the
  combined use of electro-magnetic calorimeter ($EMC$) and muon system
  ($IFR$).  The reconstruction has been validated on data control
  samples: $e^+e^- \rightarrow K^0_SK^0_L$ and $D^0 \rightarrow K^0_L
  \pi^+\pi^-$. {\bf Advisor} of graduation thesis on this subject with
  a student of ``Universit\`a degli Studi di Torino''
  (M. Pelliccioni).  The optimized $K^0_L$s have been used on many
  analysis among them the one above.
  
\item {\bf Main author} of the measurement of time-dependent $CP$
  asymmetry in $B^0 \to K^0_S K^0_S K^0_S$ decays. For this
  measurement an ad hoc $B$ production vertex measurement, based on
  $K^0_S$ vertex and $e^+e^-$ interaction point constraint, has been
  developed~\cite{Aubert:2005dy} and~\cite{Aubert:2007me}.

\item {\bf Main author} of the measurement of the branching fraction
  of $B^\pm \to \phi\pi^\pm$ and $B^0\to\phi\pi^0$ decays. They are
  expected to proceed to the very suppressed $b \to d$ elementary
  transition, and they can be used to reduce theoretical errors on $b
  \to s$ decays.  In this analysis the full particle ID informations
  from Cherenkov detector and dE/dx from silicon tracker and drift
  chamber have been used to suppress three kaon
  decays~\cite{Aubert:2007mj}.  Invited talk at ICHEP
  2006~\cite{DiMarco:2006wg,talk_ichep}.

\item {\bf Main author} of the measurement of branching fraction and
  $CP$ asymmetries of $B^0\to\phi(1020) K^0$ with the quasi-two body
  approximation~\cite{Aubert:2005ja}.  Invited talk at annual meeting
  of the {\it American Physical Society}, Division of Particles and
  Fields~\cite{conf_aps}, and WIN 2005~\cite{conf_win05}, and in two
  experimental seminars at Universit\`a di Roma ``La
  Sapienza''~\cite{seminario_roma1_1,seminario_roma1_2}.

\item {\bf Measurement of the HF production in Resistive Plate
  Chambers (RPC)}: the dissociation of Freon 134a due to gas
  discharges can produce significant concentration of HF, reducing
  the superficial resistivity with consequent efficiency loss of the
  detector. I measured the production of HF on in situ RPCs during
  BaBar data-taking in streamer and avalanche modes, interpreted the
  HF trapping mechanism and found solutions for its reduction with
  pure Argon fluxes with HV applied~\cite{Band:2008zzb}.
  
  
\item {\bf Other research activities}
  
  \begin{itemize}
    
    
  \item {\bf Determination of the tagging efficiencies and vertexing
    resolution parameters on data}, using the sample of fully
    reconstructed $B^0 \rightarrow D^{(*)-}\pi^+/\rho^+/a_1^+$
    decays. This parameterization of the vertexing algorithm has been
    used by all {\it time-dependent} $CP$ violation measurements
    presented at ICHEP 2006 conference.

  \item After the 2004 one year long shutdown, it was critical to
    recover the previous detector performances. As a member of a
    {\bf reconstruction task force} I recovered the loss in tracking
    efficiency in $K^0_S$ reconstruction. I participated to the effort
    with a study of low level tracking quantities which contributed to
    the full efficiency recovery at the beginning of the new run period.

  \end{itemize}

\item {\bf Responsibilities and contributions to detector}
  
  \begin{itemize}
    
  \item {\bf Experimental setup, data acquisition and database} 
    for the measurement of HF production in the BaBar RPCs (2006).
    
  \item {\bf Responsible for the prompt reconstruction {\it farm }}
    of the BaBar data, located in Padova (Italy), during the
    high-luminosity period (Jan 2006 - July 2006).

  \item {\bf Upgrade of the BaBar IFR forward endcap} changing the
    RPCs operating mode from {\it streamer} to {\it avalanche}: the
    much lower integrated current in the latter mode results in the
    possibility to run RPCs with a rate of 100 Hz/cm$^2$, needed
    during high luminosity period, at the cost of pre-amplified
    electronics.  After a test of few chambers, the whole endcap has
    been successfully upgraded~\cite{BandProc}.

  \item {\bf IFR Operations Manager} (March 2005 - Sep 2005): my tasks
    were maintenance and monitoring of the detectors during the data
    taking period, diagnostic and solution of malfunctioning of this
    subsystem, {\bf study of RPC aging}~\cite{Anulli:2005wi} of the
    new chambers.

  \item {\bf IFR barrel upgrade with Limited Streamer Tubes (LST)}:
    due to the efficiency degradation of barrel RPCs in the first
    data-taking period, they have been replaced with LSTs. My task
    have been the construction of the capacitive readout, detector
    installation, their first operation and data quality
    assurance.

  \end{itemize}

\end{enumerate}

\begin{center}
{ \bf Teaching Experience }

\end{center}

 \begin{itemize}  
 \item academic year 2004-2005: teaching assistant in Laboratory
   of Informatics for Physicists for undergraduate students at
   Universit\`a di Roma ``La Sapienza''.
\end{itemize}

\begin{center}
 { \bf Computing  Experience }
\end{center}

Working with $CMS$ and $BaBar$ software offered to me a broad
experience in computing:
\begin{itemize}
\item in-depth knowledge of C, C++ (reconstruction and analysis code with {\it Beta} (BaBar) and 
  {\it CMSSW} (CMS)), shell
  scripting ({\it BASH}, {\it CSH}), HTML, Python and Perl scripting.  
\item member of BaBar {\it Prompt Reconstruction Group}, responsible
  of the computing facilities for the data-taking (Padova farm).
\item offline CMS prompt reconstruction and data flow and certification
\end{itemize}

 
%%%%%%%%%%%%%%%%%%%%%%%%%%%
\newpage

\begin{center}
  {\bf Selected publications}
\end{center}

I report here a list of selected publications that I retain more
relevant, with bibliometric indices in Table~\ref{tab:biblio}.

{\bf H-Index (from spires): 66}.

%% {\bf Impact factors for year 2009 of the journals where the following
%%   papers have been published}:

%% \begin{itemize}
%% \item ``Physical Review Letters'': {\bf 7.329}
%% \item ``Physical Review D'': {\bf 3.475}
%% \item ``Physics Letters B'': {\bf 3.056}
%% \item ``Nuclear Instruments and Methods in Physics Research Section A'': {\bf 1.317}
%% \item ``Journal of instrumentation'': {\bf 2.102}
%% \end{itemize}

%-------------------------------------------------------------------------------
\begin{table}[h!]
  \begin{center}     
    \caption{ Citation summary from SPIRES \label{tab:biblio}}
\begin{tabular}{|l|c|} \hline
{\bf citation number n $>$ 500} &  {\bf 1} \\
{\bf citation number 250 $<$n$<$ 499} & {\bf 3} \\ 
citation number 100 $<$n$<$ 249 &  29 \\
citation number  50 $<$n$<$  99 &  84\\
citation number  10 $<$n$<$  49 &  257\\
citation number   1 $<$n$<$   9 &  111\\
no citations &  2 \\
\hline
{\bf Total papers} & {\bf 487} \\
\hline
\end{tabular}

  \end{center}
\end{table}

I signed all the CMS papers, and all the $BaBar$ papers since my
Ph.D. (2004) to 2009. I report in the following only the ones where 
I consider I gave a major contribution. 

\vskip 1.0 truecm

\begin{thebibliography}{99999}
  \bibitem[*]{cmsphyspapers}
        {\bf CMS physics papers}
        \\    

\bibitem[pub38]{monojet} 
  A.~M.~Sirunyan {\it et al.} [CMS Collaboration],
  ``Search for dark matter produced with an energetic jet or a hadronically decaying W or Z boson at $ \sqrt{s}=13 $ TeV,''
  JHEP {\bf 1707}, 014 (2017)

\bibitem[pub37]{photonpaper} 
  V.~Khachatryan {\it et al.} [CMS Collaboration],
  ``Performance of Photon Reconstruction and Identification with the CMS Detector in Proton-Proton Collisions at sqrt(s) = 8 TeV,''
  JINST {\bf 10}, no. 08, P08010 (2015)

\bibitem[pub36]{elepaper} 
  V.~Khachatryan {\it et al.} [CMS Collaboration],
  ``Performance of Electron Reconstruction and Selection with the CMS Detector in Proton-Proton Collisions at $\sqrt{s}$ = 8  TeV,''
  JINST {\bf 10}, no. 06, P06005 (2015)

\bibitem[pub35]{h4l_jcp} 
  V.~Khachatryan {\it et al.} [CMS Collaboration],
  ``Constraints on the spin-parity and anomalous HVV couplings of the Higgs boson in proton collisions at 7 and 8 TeV,''
  Phys.\ Rev.\ D {\bf 92}, no. 1, 012004 (2015)

\bibitem[pub34]{h4l_legacy} 
  S.~Chatrchyan {\it et al.} [CMS Collaboration],
  ``Measurement of the properties of a Higgs boson in the four-lepton final state,''
  Phys.\ Rev.\ D {\bf 89}, no. 9, 092007 (2014)

\bibitem[33]{h4l_differential} 
  V.~Khachatryan {\it et al.} [CMS Collaboration],
  ``Measurement of differential and integrated fiducial cross sections for Higgs boson production in the four-lepton decay channel in pp collisions at $ \sqrt{s}=7 $ and 8 TeV,''
  JHEP {\bf 1604}, 005 (2016)

\bibitem[pub32]{Chatrchyan:2013iaa} 
  S.~Chatrchyan {\it et al.} [CMS Collaboration],
  ``Measurement of Higgs boson production and properties in the WW decay channel with leptonic final states,''
  JHEP {\bf 1401}, 096 (2014)

\bibitem[pub31]{higgsprop}
  S.~Chatrchyan {\it et al.}  [CMS Collaboration],
  ``Study of the Mass and Spin-Parity of the Higgs Boson Candidate via Its Decays to Z Boson Pairs'',
  Phys.\ Rev.\ Lett.\  {\bf 110}, 081803 (2013)

\bibitem[pub30]{higgsdiscovery} 
  S.~Chatrchyan {\it et al.}  [CMS Collaboration],
  ``Observation of a new boson at a mass of 125 GeV with the CMS experiment at the LHC,''
  Phys.\ Lett.\ B {\bf 716}, 30 (2012)

\bibitem[pub29]{Chatrchyan:2012ty} 
  S.~Chatrchyan {\it et al.}  [CMS Collaboration],
  ``Search for the standard model Higgs boson decaying to a W pair in the fully leptonic final state in pp collisions at sqrt(s) = 7 TeV,''
  Phys.\ Lett.\ B {\bf 710}, 91 (2012)

\bibitem[pub28]{Chatrchyan:2012tx} 
  S.~Chatrchyan {\it et al.}  [CMS Collaboration],
  ``Combined results of searches for the standard model Higgs boson in pp collisions at sqrt(s) = 7 TeV,''
  Phys.\ Lett.\ B {\bf 710}, 26 (2012)

\bibitem[pub27]{Chatrchyan:2012dg} 
  S.~Chatrchyan {\it et al.}  [CMS Collaboration],
  ``Search for the standard model Higgs boson in the decay channel H to ZZ to 4 leptons in pp collisions at sqrt(s) = 7 TeV,''
  Phys.\ Rev.\ Lett.\  {\bf 108}, 111804 (2012)

\bibitem[pub26]{Chatrchyan:2011jz} 
  S.~Chatrchyan {\it et al.} [CMS Collaboration],
  ``Measurement of the lepton charge asymmetry in inclusive $W$ production in pp collisions at $\sqrt{s}$ = 7 TeV,''
  JHEP {\bf 1104}, 050 (2011)

\bibitem[pub25]{cms_hww_paper2010}
  CMS~Collaboration,
  ``Measurement of $W^+W^-$ production and search for the Higgs boson in pp collisions at $\sqrt{s}$ = 7 TeV''
  Physics Letters B {\bf 699}, 25 (2010)

\bibitem[pub24]{Chatrchyan:2011ek} 
  S.~Chatrchyan {\it et al.}  [CMS Collaboration],
  ``Inclusive search for squarks and gluinos in $pp$ collisions at $\sqrt{s}=7$ TeV,''
  Phys.\ Rev.\ D {\bf 85}, 012004 (2012)

\bibitem[pub23]{cms_vecbos_paper}
  CMS~Collaboration,
  ``Jet Production Rates in Association with W and Z Bosons in pp Collisions at sqrt(s) = 7 TeV''
  JHEP {\bf 1201}, 010 (2012)

\bibitem[pub22]{Khachatryan:2010xn} 
  V.~Khachatryan {\it et al.}  [CMS Collaboration],
  ``Measurements of Inclusive W and Z Cross Sections in pp Collisions at sqrt(s)=7 TeV,''
  JHEP {\bf 1101}, 080 (2011)

\bibitem[pub21]{inclusiveV_2010} 
  S.~Chatrchyan {\it et al.} [CMS Collaboration],
  ``Measurement of the Inclusive $W$ and $Z$ Production Cross Sections in $pp$ Collisions at $\sqrt{s}=7$ TeV,''
  JHEP {\bf 1110}, 132 (2011)


\vskip 1.0 truecm

\bibitem[*]{cmsdetpapers}
{\bf CMS detector papers}
\\

\bibitem[pub20]{cms_craft}
  CMS~Collaboration,
  ``Commissioning of the CMS experiment and the cosmic run at four tesla'',
  JINST {\bf 5} (2010) T03001

\bibitem[pub19]{ecal_craft}
 CMS~Collaboration
 ``Performance and operation of the CMS electromagnetic calorimeter''
 JINST {\bf 5} (2010) T03010

\bibitem[pub18]{time_craft}
 CMS~Collaboration
 `` Time reconstruction and performance of the CMS electromagnetic calorimeter''
 JINST {\bf 5} (2010) T03011

%\cite{:2008zzk}
\bibitem[pub17]{longo_cms}
  R.~Adolphi {\it et al.}  [CMS Collaboration],
  ``The CMS experiment at the CERN LHC,''
  JINST {\bf 3} (2008) S08004





\vskip 1.0 truecm

\bibitem[*]{babarphyspapers}
{\bf BaBar physics papers}

%\cite{Aubert:2008gy}
\bibitem[pub16]{Aubert:2008gy}
  B.~Aubert {\it et al.}  [BABAR Collaboration],
  ``Measurement of Time-Dependent CP Asymmetry in $B^0 \to K^0_{S} \pi^0
  \gamma$ Decays,''
  Phys.\ Rev.\  D {\bf 78} (2008) 071102
  %%CITATION = PHRVA,D78,071102;%%


%\cite{Aubert:2007sd}
\bibitem[pub15]{Aubert:2007sd}
  B.~Aubert {\it et al.}  [BABAR Collaboration],
  ``Measurements of CP-Violating Asymmetries in the Decay $B^0 \to K^+K^-K^0$''
  Phys.\ Rev.\ Lett.\  {\bf 99}, 161802 (2007)
%  [arXiv:0706.3885 [hep-ex]].
  %%CITATION = PRLTA,99,161802;%%

%\cite{Aubert:2007me}
\bibitem[pub14]{Aubert:2007me}
  B.~Aubert {\it et al.}  [BABAR Collaboration],
  ``Measurement of CP Asymmetries in $B^0 \to K^0_S K^0_S K^0_S$ Decays'',
  Phys.\ Rev.\  D {\bf 76}, 091101 (2007)
%  [arXiv:hep-ex/0702046].
  %%CITATION = PHRVA,D76,091101;%%

%\cite{Aubert:2007mj}
\bibitem[pub13]{Aubert:2007mj}
  B.~Aubert {\it et al.}  [BABAR Collaboration],
  ``Observation of CP violation in $B^0 \to K^{+} \pi^{-}$ and $B^0 \to \pi^{+} \pi^{-}$,''
  Phys.\ Rev.\ Lett.\  {\bf 99}, 021603 (2007)
%  [arXiv:hep-ex/0703016].
  %%CITATION = PRLTA,99,021603;%%

%\cite{Aubert:2007mgb}
\bibitem[pub12]{Aubert:2007mgb}
  B.~Aubert {\it et al.}  [BABAR Collaboration],
  ``Measurement of the CP-violating asymmetries in $B^0 \to K^0_{S} \pi^0$ and of the branching fraction of $B^0 \to K^0 \pi^0$,''
  Phys.\ Rev.\  D {\bf 77}, 012003 (2008)
  %%[arXiv:0707.2980 [hep-ex]].
  %%CITATION = PHRVA,D77,012003;%%

\bibitem[pub11]{Aubert:2007hm} 
  B.~Aubert et al. [BABAR Collaboration] 
  {\it ``Improved measurement of CP violation in neutral B decays to $c \bar{c} s$''},
  Phys.\ Rev.\ Lett.\  {\bf 99}, 171803 (2007)
 % Arxiv:hep-ex/0703021.

%\cite{Aubert:2006nn}
\bibitem[pub10]{Aubert:2006nn}
  B.~Aubert {\it et al.}  [BABAR Collaboration],
  ``Search for $B^+ \to \phi \pi^+$ and $B^0 \to \phi \pi^0$ decays,''
  Phys.\ Rev.\ D {\bf 74}, 011102 (2006)
%  [arXiv:hep-ex/0605037].
  %%CITATION = HEP-EX 0605037;%%

%\cite{Aubert:2006nu}
\bibitem[pub9]{Aubert:2006nu}
  B.~Aubert {\it et al.}  [BABAR Collaboration],
  ``Dalitz plot analysis of the decay $B^{\pm} \to K^{\pm} K^{\pm} K^{\mp}$,''
  Phys.\ Rev.\  D {\bf 74}, 032003 (2006)
%  [arXiv:hep-ex/0605003].
  %%CITATION = PHRVA,D74,032003;%%


%\cite{Aubert:2006wv}
\bibitem[pub8]{Aubert:2006wv}
  B.~Aubert {\it et al.}  [BABAR Collaboration],
  ``Observation of CP violation in $B^0 \to \eta^\prime K^0$ decays,''
  Phys.\ Rev.\ Lett.\  {\bf 98}, 031801 (2007)
  %%[arXiv:hep-ex/0609052].
  %%CITATION = PRLTA,98,031801;%%

%\cite{Aubert:2006zy}
\bibitem[pub7]{Aubert:2006zy}
  B.~Aubert {\it et al.}  [BABAR Collaboration],
  ``Search for the decay $B^0 \to K^0_{s} K^0_{s} K^0_{L}$,''
  Phys.\ Rev.\  D {\bf 74}, 032005 (2006)
  %%[arXiv:hep-ex/0606031].
  %%CITATION = PHRVA,D74,032005;%%

%\cite{Aubert:2006gm}
\bibitem[pub6]{Aubert:2006gm}
  B.~Aubert {\it et al.}  [BABAR Collaboration],
  ``Observation of $B^{+} \to \bar{K}^0 K^{+}$ and $B^0 \to K^0 \bar{K}^0$,''
  Phys.\ Rev.\ Lett.\  {\bf 97}, 171805 (2006)
  %%[arXiv:hep-ex/0608036].
  %%CITATION = PRLTA,97,171805;%%

%\cite{Aubert:2005ja}
\bibitem[pub5]{Aubert:2005ja}
  B.~Aubert {\it et al.}  [BABAR Collaboration],
   ``Measurement of CP asymmetries in $B^0 \to \phi K^0$ and $B^0 \to K^+ K^-K^0_S$  decays,''
  Phys.\ Rev.\ D {\bf 71}, 091102 (2005)
%  [arXiv:hep-ex/0502019].
  %%CITATION = HEP-EX 0502019;%%

%\cite{Aubert:2005dy}
\bibitem[pub4]{Aubert:2005dy}
  B.~Aubert {\it et al.}  [BABAR Collaboration],
  ``Branching fraction and CP asymmetries of $B^0 \to K^0_S K^0_S K^0_S$'', 
  Phys.\ Rev.\ Lett.\  {\bf 95}, 011801 (2005)
%  [arXiv:hep-ex/0502013].
  %%CITATION = HEP-EX 0502013;%%


\vskip 1.0 truecm

\bibitem[*]{babardetpapers}
{\bf BaBar detector papers}

%\cite{Anulli:2005wi}
\bibitem[pub3]{Anulli:2005wi}
  F.~Anulli {\it et al.},
  ``Performance of 2nd generation BaBar resistive plate chambers,''
  Nucl.\ Instrum.\ Meth.\ A {\bf 552}, 276 (2005)

%\cite{Band:2008zzb}
\bibitem[pub2]{Band:2008zzb}
  H.~R.~Band {\it et al.},
  ``Study Of HF Production In BaBar Resistive Plate Chambers'',
  Nucl.\ Instrum.\ Meth.\  A {\bf 594}, 33 (2008).

\bibitem[pub1]{BandProc}
  H.~R.~Band {\it et al.},
  ``Performance and Aging Studies of BaBar Resistive Plate Chambers''
  Nucl.\ Physics B {\bf 158}, 139-142 (2006).

\vskip 1.0 truecm

\bibitem[*]{otherpapers}
{\bf Other papers}

\bibitem[opub2]{pinci_jinst}
  C.~Antochi~Vasile {\it et al.},
  ``Combined readout of a triple-GEM detector''
  Submitted to JINST (2018)

\bibitem[opub1]{h4l_8d}
  Y.~Chen, E.~Di Marco, J.~Lykken, M.~Spiropulu, R.~Vega-Morales and S.~Xie,
  ``8D likelihood effective Higgs couplings extraction framework in $h \to 4\ell$,''
  JHEP {\bf 1501}, 125 (2015)



\vskip 1.0 truecm

  
  \begin{center}
  \bibitem{crscelti}
    {\bf Other publications and selected conference reports }
    \\
  \end{center}
  \bibitem[*]{cmsproc}
{\bf CMS proceedings and relevant public notes}
\\

\bibitem[p9]{proc_ichep12} 
  E.~Di Marco [CMS Collaboration], 
  ``Search for SM Higgs decaying to WW $\to\ell\nu\ell\nu$ and $\ell\nu$qq at CMS'', 
  PoS ICHEP {\bf 2012}, 076 (2013)

\bibitem[p8]{proc_ifae2012}
  E.~Di Marco [CMS Collaboration],
  ``Searches for the standard model Higgs boson at CMS'',  
  Il Nuovo Cimento {\bf 36}, 1 (2013)

\bibitem[p7]{Chatrchyan:2013oev} 
  S.~Chatrchyan {\it et al.}  [ CMS Collaboration],
  ``Measurement of W+W- and ZZ production cross sections in pp collisions at sqrt(s)=8 TeV,''
  CERN-PH-EP-2012-376, arXiv:1301.4698 [hep-ex].
  %%CITATION = ARXIV:1301.4698;%%
%\cite{:2013qh}

\bibitem[p6]{proc_plhc} 
  E.~Di Marco [CMS Collaboration],
  ``Observation of W and Z boson candidates with the CMS experiment'',
  DESY-PROC-2010-01.
  %%CITATION = DESY-PROC-2010-01;%%
%\cite{Chatrchyan:2011dx}

\bibitem[p5]{proc_ifae} E.~Di Marco ``Measurement of W and Z
  production in association with jets with CMS detector'' Il Nuovo
  Cimento {\bf 32}, 3 (2009)

\bibitem[p4]{proc_siena}
  E.~Di Marco,
  ``The CMS ECAL data quality monitoring and first results with cosmics data'',
  Nucl.\ Physics B {\bf 197}, 267-270 (2009)



\bibitem[*]{babarproc}
{\bf BaBar proceedings and relevant public notes}

%\cite{Bona:2007qt}
\bibitem[p3]{Bona:2007qt}
  M.~Bona {\it et al.},
  ``SuperB: A High-Luminosity Asymmetric e+ e- Super Flavor Factory. Conceptual Design Report,''
  SLAC-R {\bf 856}, INFN-AE {\bf 07-02}, LAL {\bf 07-15}
  %%CITATION = ARXIV:0709.0451;%%

%\cite{DiMarco:2006wg}
\bibitem[p2]{DiMarco:2006wg}
  E.~Di Marco  [BABAR Collaboration],
  ``Measurement of direct CP asymmetries in charmless hadronic B decays,''
  Moscow 2006, ICHEP 843-850 

%\cite{DiMarco:2007af}
\bibitem[p1]{DiMarco:2007af}
  E.~Di Marco,
  ``Measurement of angle $\beta$ with time-dependent CP asymmetry in $B^0 \to K^+ K^-K^0$ decays,''
  proceeding for an invited talk at 4th International Workshop on the 
  CKM Unitarity Triangle (CKM 2006), Nagoya, Japan, 12-16 Dec 2006.







  \newpage

  \begin{center}
  \bibitem{notecms}
    {\bf CMS public notes }
    \\
  \end{center}
  \bibitem[np15]{pas_HZZProp} ``On the mass and spin-parity of the Higgs boson candidate via its decays to Z boson pairs''
CMS Paper: HIG-12-041, Accepted by Phys. Rev. Lett., 2012

\bibitem[np14]{pas_HZZ2} ``Updated results on the new boson discovered in the search for the standard model Higgs boson in the ZZ to 4 leptons channel in pp collisions at $\sqrt{s}$ = 7 and 8 TeV'', 
CMS PAS: HIG-12-041, 2012

\bibitem[np13]{pas_HWW5} ``Search for the Higgs Boson in the Fully Leptonic $W^+W^-$ Final State (12.1 fb$^{-1}$)''
CMS PAS: HIG-12-042 ``Evidence for a new state in the search for the standard model Higgs boson in the $H\to ZZ\to 4l$ channel in pp collisions at $\sqrt{s}$ = 7 and 8 TeV''
$CMS$ Physics Analysis Summary, 2012

\bibitem[np12]{pas_WW1} ``Measurement of WW production rate (3.54 $fb^{-1}$ at $\sqrt{s}$=8 TeV)''
CMS PAS: SMP-12-013, 2012

\bibitem[np11]{pas_HWWICHEP12} ``Search for the standard model Higgs boson decaying to W+W− in the fully leptonic final state in pp collisions at $\sqrt{s}$ = 8 TeV''
CMS PAS: HIG-12-017, 2012

\bibitem[np10]{pas_HZZICHEP12} ``Evidence for a new state in the search for the standard model Higgs boson in the $H \to ZZ \to 4\ell$ channel in pp collisions at $\sqrt{s}$ = 7 and 8 TeV''
CMS PAS: HIG-12-016, 2012

\bibitem[np9]{pas_HiggsPP} ``Search for a fermiophobic Higgs boson in pp collisions at $\sqrt{s}$ = 7 TeV''
CMS PAS: HIG-12-009, Submitted to JHEP, 2012

\bibitem[np8]{pas_WW2} ``Measurement of WW production rate (5.1 $fb^{-1}$ at $\sqrt{s}$=7 TeV)''
CMS PAS: SMP-12-005, 2012

\bibitem[np7]{pas_HWW3} ``Search for the Higgs Boson in the Fully Leptonic $W^+W^-$ Final State (4.6 fb$^{-1}$)''
CMS PAS: HIG-11-024 ``Search for the Higgs Boson in the Fully Leptonic $W^+W^-$ Final State''
$CMS$ Physics Analysis Summary, 2011

\bibitem[np6]{pas_recoelectrons}
CMS PAS: EGM-09-001 ``Electron Reconstruction in CMS''
$CMS$ Physics Analysis Summary, 2009

\bibitem[np5]{pas_idelectrons}
CMS PAS: EGM-09-002 ``Electron Identification in CMS''
$CMS$ Physics Analysis Summary, 2009

\bibitem[np4]{pas_Wjets}
CMS PAS: EWK-08-006 ``Study of the ratio of W+jets to Z+jets in pp collisions at sqrt(s) =  10 TeV with the CMS detector at the CERN LHC ''
$CMS$ Physics Analysis Summary, 2008

\bibitem[np3]{pas_Zjets}
CMS PAS: EWK-08-006 ``Study of Z production in association with jets in pp collisions at sqrt(s) = 10 TeV with the CMS detector at the CERN LHC''
$CMS$ Physics Analysis Summary, 2008

\bibitem[np2]{pas_HWW2}
CMS PAS: HIG-08-006 ``Search strategy for a Standard Model Higgs boson decaying in two W bosons in the fully leptonic final state''
$CMS$ Physics Analysis Summary, 2008

\bibitem[np1]{pas_HWW1}
CMS PAS: HIG-07-001 ``Search for the Higgs boson in the WW(*) decay channel with the CMS experiment''
$CMS$ Physics Analysis Summary, 2007


  \begin{center}
  \bibitem{conf}
    {\bf International Conferences talks}
    \\
  \end{center}
  \bibitem[ci9]{conf_ichep12}
  {\bf ``Search for Higgs in WW decays at CMS''}
  talk at conference ``ICHEP 2012'', Melbourne (Australia), July 2012.

\bibitem[ci8]{conf_hh}
  {\bf ``Higgs into WW and ZZ at CMS''}
  talk at conference ``Higgs Hunting 2011'', Paris (Fance), July 2011.

\bibitem[ci7]{conf_lishep}
  {\bf ``V+ Jets and V+gamma at the LHC''}
  talk at conference ``Lishep 2011'', Rio de Janeiro (Brasil), July 2011.

\bibitem[ci6]{conf_plhc}
  {\bf ``Observation of W and Z production with CMS experiment'' }
  talk at conference ``Physics at LHC 2010'' Hamburg (Germany), June 2010.

\bibitem[ci5]{conf_siena}
  {\bf ``The CMS ECAL data quality monitoring and first results with cosmics data''}
  talk at conference ``$11^{th}$ Topical Seminar on Innovative Particle and Radiation Detectors'',
  Siena (Italy), October 2008.

\bibitem[ci4]{talk_ichep}
  {\bf ``Direct CP Asymmetries in Charmless B Decays with the BaBar experiment''}
  International Conference on High Energy Physics (ICHEP2006), Moscow (Russia), July 2006
  \vspace{3mm}

\bibitem[ci3]{talk_nagoya}
  {\bf ``Measurement of CKM angle $\beta$ with time dependent Dalitz plot analysis of 
    $B^0\rightarrow K^+K^-K^0$ decays''}
  IV CKM Workshop, Nagoya (Giappone), December 2006

\bibitem[ci2]{conf_win05}
  {\bf ``Measurement of $\sin 2\beta$ at B-factories''}
  Worksop Weak Interactions and Neutrinos (WIN 2005), Delphi (Grecia), June 2005
  \vspace{3mm}

\bibitem[ci1]{conf_aps}
  {\bf ``Measurement of CP asymmetries in $b \rightarrow s$ decays at $BaBar$''}
  Annual Meeting of the American Physical Society, Tampa (FL), April 2005
  \vspace{3mm}




  \begin{center}
  \bibitem{confita}
    {\bf Italian Conferences talks}
    \\
  \end{center}
  \bibitem[cn3]{conf_ifae12}
  {\bf ``Ricerca di Higgs a CMS''}
  plenary talk at conference ``Incontri di Fisica delle Alte Energie - XI edizione'',
  Ferrara (Italy), April 2012.

\bibitem[cn2]{conf_ifae09}
  {\bf ``Misura di W e Z con produzione associata di jet a CMS''}
  talk at conference ``Incontri di Fisica delle Alte Energie - VIII edizione'',
  Bari (Italy), April 2009.

\bibitem[cn1]{talk_lnf}
  {\bf ``Measurement of CP asymmetry in $b \rightarrow s$ decays''},
  plenary talk the the ``LNF spring school'', Roma (Italy), April 2004
  \vspace{3mm}

  
  \begin{center}
  \bibitem{seminari}
    {\bf Seminars}
    \\
  \end{center}
    % seminari
\bibitem[s5]{seminario_caltech}
  {\bf ``Higgs searches with diboson decays with CMS experiment''}
  High Energy Phyisics Seminar given at California Institute of Technology, Pasadena (USA), November 2011

\bibitem[s4]{seminario_ucsd}
  {\bf ``Search for new physics with time-dependent CP asymmetries in $b\to s$ transitions''}
  High Energy Phyisics Seminar given at University of California, San Diego (USA), December 2006

\bibitem[s3]{seminario_princeton}
  {\bf ``Search for new physics with time-dependent CP asymmetries in $b\to s$ transitions''}
  High Energy Phyisics Seminar given at Princeton University (USA), January 2007

\bibitem[s2]{seminario_roma1_2}
  {\bf ``Search for physics beyond the Standard Model at B-factories''}
  Seminar of Particles and Fields,  Universit\`a di Roma ``La Sapienza'', Roma (Italy), May 2006
  \vspace{3mm}

\bibitem[s1]{seminario_roma1_1}
  {\bf ``Flavour Changing Neutral Current processes in $B$ decays''}
  Seminar of Particles and Fields, University of Rome ``La Sapienza'', Roma (Italy), November 2004
  \vspace{3mm}


  \newpage

  \begin{center}
  \bibitem{notecms}
    {\bf CMS internal notes }
    \\
  \end{center}
  \bibitem[*]{notesHiggs}
        {\bf Higgs physics notes}
        \\    

\bibitem[in58]{AN_2013_022}
CMS AN-2013/022: ``Higgs Boson Decaying to WW in the Leptonic Final State using 2011 and 2012 Data'', Feb. 2013

\bibitem[in57]{AN_2013_002}
CMS AN-2013/022: ``Higgs Boson Decaying to WW in the Leptonic Final State Measurement Using Razor Variables'', Jan. 2013

\bibitem[in56]{AN_2013_378}
CMS AN-2013-378: ``Higgs Boson Decaying to WW in the Leptonic Final State'', Nov. 2012

\bibitem[in55]{AN_2012_346}
CMS AN-2012-346: ``Observation at 4.5 standard deviations of a Standard Model Higgs-like boson with mass mH = 126.2 +- 0.7GeV,using H to ZZ to 4l events with 17.3 /fb of Data at sqrt(s) = 7 and 8 TeV'', Nov. 2012

\bibitem[in54]{AN_2012_194}
CMS AN-2012-194: ``Search for the Higgs Boson Decaying to WW in the Fully Leptonic Final State at 8 TeV'', July 2012

\bibitem[in53]{AN_2012_202}
CMS AN-2012-202: ``Search for the Standard Model Higgs boson in the decay channel $H\to ZZ\to 4$ with 10.3 fb$^{-1}$ of Data'', July 2012

\bibitem[in52]{AN_2012_310}
CMS AN-2012-310: ``Spin Measurements of the new Boson observed around 125 GeV wit $H\to\gamma\gamma$, Aug. 2012

\bibitem[in51]{AN_2012_121}
CMS AN-2012-121: ``Optimization of Muon Isolation to reduce W+Jets Background and mitigate Pile-up effects in Di-Lepton final states'', July 2012

\bibitem[in50]{AN_2012_086}
CMS AN-2012-086: ``Study of associated Higgs (VH) Production with $H\to WW\to 2\ell 2\nu$ and hadronic V decay with 4.9 fb$^{-1}$ at 7 TeV'', May 2012

\bibitem[in49]{AN_2012_071}
CMS AN-2012-071: ``Search for the Higgs Boson in $H\to WW\to 2\ell 2\nu$ in pp Collisions at $\sqrt{s}$=7 TeV with razor variables'', Feb. 2012

\bibitem[in48]{AN_2011_463}
CMS AN-2011-463: ``Shape Analyses and Systematic Uncertainties for the Search of Higgs Boson Decaying to WW in the Fully Leptonic Final State with 4.6 fb$^{-1}$ at 7 TeV''

\bibitem[in47]{AN_2011_460}
CMS AN-2011-460: ``Search for the Higgs Boson Decaying to WW in the Fully Leptonic Final State with 4.6 /fb at 7 TeV'', Nov. 2011

\bibitem[in46]{AN_2011_432}
CMS AN-2011-432: ``Search for Higgs Boson Decays to Two W Bosons in the Fully Leptonic Final State with Full 2011 pp Dataset at $\sqrt{s}$ = 7 TeV'', Dec. 2011

\bibitem[in45]{AN_2011_364}
CMS AN-2011-364: ``Search for the Higgs Boson Decaying to WW in the Fully Leptonic Final State. Update with Lepton Photon 2011 dataset'', Aug. 2011

\bibitem[in44]{AN_2011_148}
CMS AN-2011-148: ``Search for Higgs Boson Decays to Two W Bosons in the Fully Leptonic Final State $\sqrt{s}$ = 7 TeV with 2011 data of CMS detector'', Aug. 2011

\bibitem[in43]{AN_2011_201}
CMS AN-2011-201: ``Search for the Higgs Boson in the Fully Leptonic WW Final State'', June 2011

\bibitem[in42]{AN_2010_411}
CMS AN-2010-411: ``Search for Higgs Boson Decays to Two W Bosons in the Fully Leptonic Final State $\sqrt{s}$ = 7 TeV'', March 2011

\bibitem[in41]{AN_2010_139}
CMS AN-2010-139: ``Search Strategy for a Standard Model Higgs Boson Decaying to Two W Bosons in the Fully Leptonic Final State at $\sqrt{s}$ = 10 TeV'', Aug. 2009

\bibitem[in40]{AN_2009_020}
CMS AN-2009-020: ``Projected exclusion limits on the SM Higgs boson cross sections obtained by combining the $H \to WW^(*)$ and ZZ$^(*)$ decay channels'', March 2009

\bibitem[in39]{AN_2008_039}
CMS AN-2008-039: ``Search Strategy for a Standard Model Higgs Boson Decaying to Two W Bosons in the Fully Leptonic Final State'', May 2009

\bibitem[in38]{AN_2007_037}
CMS AN-2007-037: ``Search for the Higgs boson in the WW$^(*)$ decay channel with the CMS experiment'', Jan. 2008



\vskip 1.0 truecm

\bibitem[*]{notesSUSY}
        {\bf SUSY physics notes}
        \\    

\bibitem[in37]{AN_2012_279}
CMS AN-2012-279: ``Search for SUSY events with l l+bjets at 8 TeV with the Razor'', Jul. 2012

\bibitem[in36]{AN_2012_278}
CMS AN-2012-278: ``Search for stop direct production at 8 TeV with the Razor'', Feb. 2013

\bibitem[in35]{AN_2012_234}
CMS AN-2012-234: ``Razor Inclusive Search for SUSY at 8 TeV'', Jan. 2013

\bibitem[in34]{AN-2011-516}
CMS AN-2011-516: ``An inclusive search for the decays of a new high-mass state with MET in multijet events using the Razor variables'', June 2012

\bibitem[in33]{AN_2011_458}
CMS AN-2011-458: ``Search for supersymmetry in final states with b-jets using the razor variables'', Apr. 2012

\bibitem[in32]{AN_2011_233}
CMS AN-2011-233: ``Razor inclusive search for pair-produced heavy particles at $\sqrt{s}$=7 TeV'', July 2011

\bibitem[in31]{AN_2011_219}
CMS AN-2011-219: ``2011 Razor Trigger Suite'', June 2011

\bibitem[in30]{AN_2010_288}
CMS AN-2010-288: ``Inclusive search for pair-produced heavy particles at $\sqrt{s}$=7 TeV'', March. 2011





\vskip 1.0 truecm

\bibitem[*]{notesEWK}
        {\bf Electroweak physics notes}
        \\    

\bibitem[in29]{AN_2012_205}
CMS AN-2012-205: ``Measurement of the production cross section of pp $\to$ WW at 8 TeV'', May 2012

\bibitem[in28]{AN_2012_328}
CMS AN-2012-328: ``W+jets and Z+jets studies with 5 fb$^{-1}$ of proton-proton collisions at $\sqrt{s}$=8 TeV'', Sep. 2012

\bibitem[in27]{AN_2012_163}
CMS AN-2012-163: ``Measurement of the production cross section ratio of WW to Z at 7 TeV'', May 2012

\bibitem[in26]{AN_2012_036}
CMS AN-2012-036: ``WW cross section measurement in the Fully Leptonic Final State with 4.63 fb$^{-1}$ at 7 TeV'', May 2012

\bibitem[in25]{AN-2011-508}
CMS AN-2011-508: ``W+jets and Z+jets studies with 2011A data'', Dec. 2011

\bibitem[in24]{AN_2011_162}
CMS AN-2011-162: ``Measurement of the electron charge asymmetry in $W\to e\nu$ + jets in proton-proton collisions at $\sqrt{s}$=7TeV'', June 2011

\bibitem[in23]{AN_2010_461}
CMS AN-2010-461: ``Substracting the background from top quark production in the W$\to\ell\nu$; + jets measurements'', Feb. 2011

\bibitem[in22]{AN_2010_425}
CMS AN-2010-425: ``Measurement of the $W\to e\nu$ to $Z\to ee$+ jets ratio in proton--proton collisions at $\sqrt{s}$=7 TeV'', March 2011

\bibitem[in21]{AN_2010_413}
CMS AN-2010-413: ``Measurement of the Associated Production of Vector Bosons and Jets in proton-proton collisions at $\sqrt{s}$ = 7 TeV'', March 2011

\bibitem[in20]{AN_2010_384}
CMS AN-2010-384: ``Measurement of Berends-Giele Scaling and Z+jets/W+jets cross-section ratio with muon final states using full Particle-Flow analysis'', May 2011

\bibitem[in19]{AN_2010_344}
CMS AN-2010-344: ``First Measurement of pp $\to$ WW Production Cross-Section at $\sqrt{s}$ = 7 TeV'', Feb. 2011

\bibitem[in18]{AN_2010_264}
CMS AN-2010-264: ``Updated Measurements of the Inclusive W and Z Cross Sections at 7 TeV'', Nov. 2010

\bibitem[in17]{AN_2010_136}
CMS AN-2010-136: ``Study of Production of Vector Bosons and Jets at 7 TeV for ICHEP2010 Conference'', Jan. 2011

\bibitem[in16]{AN_2009_092}
CMS AN-2009-092: ``The Z+jets candle in dielectron+jets and dimuon+jets final states at CMS with pp collisions at center-of-mass energy 10 TeV'', June 2009

\bibitem[in15]{AN_2009_045}
CMS AN-2009-045: ``The W+jets/Z+jets Ratio at CMS with pp collisions at center-of-mass energy 10 TeV'', June 2009

\bibitem[in14]{AN_2008_105}
CMS AN-2008-105: ``The Ratio of $W\to\mu\nu$+ N jets to $Z\to\mu\mu$+N jets Versus N'', Nov. 2008

\bibitem[in13]{AN_2008_096}
CMS AN-2008-096: ``The Ratio of $W\to e\nu$+ N jets to $Z\to ee$+N jets Versus N'', Nov. 2008

\bibitem[in12]{AN_2008_095}
CMS AN-2008-095: ``The $Z\to\mu\mu$ + jets data candle'', Nov. 2008

\bibitem[in11]{AN_2008_092}
CMS AN-2008-092: ``The $Z\to ee$ +jets data candle'', Nov. 2008

\bibitem[in10]{AN_2008_091}
CMS AN-2008-091: ``ALPGEN VB+jets Validation Studies'', Oct. 2008





\vskip 1.0 truecm

\bibitem[*]{notesEGamma}
        {\bf ECAL and e/$\gamma$ notes}
        \\    
\bibitem[in9]{AN_2012/408}
CMS AN-2012/048: ``CMS ECAL Performance with Z Dielectron Events in 2012 Data'', Nov. 2012

\bibitem[in8]{AN_2012_327}
CMS AN-2012/327: ``Electron Energy Reconstruction Using a Multivariate Regression'', Sep. 2012

\bibitem[in7]{AN_2012_311}
CMS AN-2012-311: ``Electron Efficiency Measurement Using the Radiative $Z\to ee\gamma$ Sample'', Oct. 2012

\bibitem[in6]{AN_2010_443}
CMS AN-2010-443: ``Calibration of the ECAL energy scale using the J/$\Psi$; resonance'', March 2012

\bibitem[in5]{AN_2010_235}
CMS AN-2010-235: ``Electron Commissioning in CMS from first 7TeV Minimum Bias Data'', March 2011

\bibitem[in4]{AN_2010_307}
CMS AN-2010-307: ``Likelihood algorithm for electron identification at $\sqrt{s}$=7~TeV'', Oct. 2010

\bibitem[in3]{AN_2010_306}
CMS AN-2010-306: ``Electron identification efficiency measurements using electrons from W'', Oct. 2010

\bibitem[in2]{AN_2009_164}
CMS AN-2009-164: ``Electron Reconstruction in CMS'', Jan. 2010

\bibitem[in2]{AN_2009_178}
CMS AN-2009-178: ``Electron Identification in CMS'', Nov. 2009

\bibitem[in1]{AN_2008_110}
CMS AN-2008-110: ``Electron identification in the CMS experiment based on a likelihood algorithm'', Dec. 2008

  


%  \begin{center}
%  \bibitem{seminari}
%    {\bf Full list of publications}
%  \end{center}
%  \input{pub_all}

\end{thebibliography}



\end{document}


