\documentclass[11pt,twoside,a4paper]{article}
\usepackage{geometry}
\geometry{a4paper, top=2.3cm, bottom=2.3cm, left=2.3cm, right=2.3cm}
\usepackage{graphicx} 
\usepackage{xcolor}
\usepackage{xspace}
\usepackage{url}
\usepackage{amsmath}
\usepackage{cite}
\usepackage[T1]{fontenc}
\newcommand{\HRule}{\rule{\linewidth}{0.2mm}}
\thispagestyle{empty}
\setlength{\parindent}{0mm}
\setlength{\parskip}{0mm}
%%%%%%%%%%%%%%%%%%%%%%%%%%%
%% \hoffset = -2.0 cm
%% \voffset = -3.0 cm
%% \textheight = 25.0 cm
%% \marginparwidth = 1.0 cm
%% \evensidemargin = 1.0 cm
%% \textwidth =16.1 cm
%% \renewcommand{\arraystretch}{1.5}
%%%%%%%%%%%%%%%%%%%%%%%%%%%

%%% remove "References" title
\usepackage{etoolbox}
\patchcmd{\thebibliography}{\section*{\refname}}{}{}{}
%%%%%%%%%%%%%%%%%%%%%%%%%%%%%%


\begin{document}

\begin{center}
\textbf{Contributi all’organizzazione di eventi di comunicazione della scienza}
\end{center}
\vskip 0.3 truecm

Partecipazione e sviluppo delle prime tre edizioni del progetto
\textit{Lab2Go}, progetto di alternanza scuola-lavoro, che intende
perseguire la catalogazione e documentazione degli strumenti
scientifici per esperimenti di fisica nelle scuole di secondo grado di
Roma e dintorni. Ho partecipato all'ideazione del progetto e
all'organizzazione della sua attuazione pratica con la logistica degli
incontri nelle scuole e ai laboratori di informatica dell'Universit\`a
Sapienza di Roma e con attivit\`a di tutoraggio in alcune scuole di
Roma, dove ho contribuito a rivalutare i piccoli laboratori delle
scuole, anche con l'attuazione di semplici esperimenti di meccanica,
elettrodinamica e ottica.

\vskip 1.0 truecm
\begin{center}
\textbf{Attivit\`a di collaborazione con le universit\`a}
\end{center}
\vskip 0.3 truecm

Ho collaborato ad attivit\`a di insegnamento di corsi universitari:
\begin{itemize}
\item 2004--2005 e 2018--2019: esercitazioni del corso ``Laboratorio
  di calcolo'' del primo anno del corso triennale di laurea in fisica,
  con il Prof. G. Organtini (Sapienza, Universit\`a di Roma)
\item 2019--: esercitazioni e tutoraggio del corso ``Physics
  Laboratory II'' del primo anno del curriculum di fisica delle
  particelle e astroparticelle del corso magistrale, con il
  Prof. G. Cavoto (Sapienza, Universit\`a di Roma)
\end{itemize}
\vskip 0.3 truecm

Ho la seguente esperienza come relatore interno o esterno di studenti di laurea o dottorato:
\begin{itemize}
\item 2015--: membro di numerose commissioni di laurea magistrale
    presso Sapienza Universit\`a di Roma
\item 2016--2019: co-relatore di una tesi di dottorato del gruppo di
  CMS Roma (argomento: sezioni d'urto differenziali del bosone W a
  LHC)
\item 2016--2017: co-relatore di due tesi di laurea del gruppo CMS
  Trieste (argomenti: sezioni d'urto inclusive differenziali e
  unfolding dello spettro in $p_T$ del bosone Z a LHC)
\item 2015: co-relatore di una tesi di laurea del gruppo di CMS Roma
  (argomento: materia oscura con canali monojet in CMS)
\item 2011--2014: relatore di 4 studenti di laurea o dottorato del
  gruppo di Caltech (argomenti su ricerche di SUSY con variabili
  cinematiche innovative, misura degli accoppiamenti anomali del
  bosone di Higgs nel canale 4 leptoni)
\item 2012: esaminatore nella commissione di dottorato all'Universidad
  de Cantabria (Santander)
\item 2011: co-relatore di uno studente di laurea del gruppo di CMS
  Trieste (argomento: sezione d'urto di produzione di bosoni Z in
  associazione con jet adronici)
\item 2005: co-relatore di una tesi di laurea del gruppo BaBar di
    Torino
\end{itemize}
\vskip 0.3 truecm

Ho partecipato come insegnante alle scuole:
\begin{itemize}
\item \textit{CMS Data Analysis School} del CERN (2013)
  sull'argomento: identificazione degli elettroni
\item \textit{CMS Data Analysis School} di Bari (2015) sull'argomento:
  calibrazione di ECAL
\item \textit{CMS Physics Objects School} di Bari (2017)
  sull'argomento: ricostruzione e calibrazione di ECAL e
  identificazione di elettroni e fotoni
\end{itemize}

\end{document}
